\documentclass[
    % -- opções da classe memoir --
    12pt,               % tamanho da fonte
    openright,          % capítulos começam em pág ímpar (insere página vazia caso preciso)
    %twoside,            % para impressão em verso e anverso. Oposto a oneside
    oneside,
    a4paper,            % tamanho do papel. 
    % -- opções da classe abntex2 --
    %chapter=TITLE,     % títulos de capítulos convertidos em letras maiúsculas
    %section=TITLE,     % títulos de seções convertidos em letras maiúsculas
    %subsection=TITLE,  % títulos de subseções convertidos em letras maiúsculas
    %subsubsection=TITLE,% títulos de subsubseções convertidos em letras maiúsculas
    % para pacote url reconhecer hifens como separador
    hyphens,
    paginasA3,  % indica que vai utilizar paginas em A3 
    GLOSSARIO, % gerar glossario a partir do arquivo defs-glossario.tex
    TODO, % indica que deve apresentar lista de pendencias 
    % -- opções do pacote babel --
    english,            % idioma adicional para hifenização
    brazil           % o último idioma é o principal do documento
    ]{ifsp-spo-inf-ctds} % ajustar de acordo com o modelo desejado para o curso

% ---
% Informações de dados para CAPA e FOLHA DE ROSTO
% ---
\titulo{STUDY FLOW}

% Trabalho em Equipe
% ver também https://github.com/abntex/abntex2/wiki/FAQ#como-adicionar-mais-de-um-autor-ao-meu-projeto
\renewcommand{\imprimirautor}{
\begin{tabular}{lr}
Kevin Klein & SP3096289 \\
Luiz Fernando & SP3096301\\
Ruan de Souza  & SP3069672 \\
Pedro Dias & SP3099211 \\
\end{tabular}
}


\disciplina{PI1A5 - Projeto Integrado I}

\preambulo{Documento final da aplicação.}

\data{2024}

% Definir o que for necessário e comentar o que não for necessário
% Utilizar o Nome Completo, abntex tem orientador e coorientador
% então vão ser utilizados na definição de professor
\renewcommand{\orientadorname}{Professor:}
\orientador{JOHNATA SOUZA SANTICIOLI}


% ---


% informações do PDF
\makeatletter
\hypersetup{
        %pagebackref=true,
        pdftitle={\@title}, 
        pdfauthor={\@author},
        pdfsubject={\imprimirpreambulo},
        pdfcreator={LaTeX with abnTeX2 using IFSP model},
        pdfkeywords={abnt}{latex}{abntex}{abntex2}{IFSP}{\ifspprefixo}{trabalho acadêmico}, 
        colorlinks=true,            % false: boxed links; true: colored links
        linkcolor=blue,             % color of internal links
        citecolor=blue,             % color of links to bibliography
        filecolor=magenta,              % color of file links
        urlcolor=blue,
        bookmarksdepth=4
}
\makeatother
% --- 

% carregando aqui referencias quando utilizando BIBLATEX
\IfPackageLoaded{biblatex}{%
\addbibresource{referencias.bib}
}{}

% ----
% Início do documento
% ----
\begin{document}



% Retira espaço extra obsoleto entre as frases.
\frenchspacing 

%somente para o exemplo, fica primeiro
% \todo[inline]{Remover texto informativo inicial}
% \input{00-info}

% -- lista de pendencias gerada pelo todonotes
% -- altere opções do usepackage para remover na versão final....
% \listoftodos
% \todo[inline]{remover lista de todo da versão final...}
\newpage

% ----------------------------------------------------------
% ELEMENTOS PRÉ-TEXTUAIS
% ----------------------------------------------------------
\pretextual

% ---
% Capa
% ---
\imprimircapa

\newcounter{todocounter}
\newcommand{\todonum}[2][]
{\stepcounter{todocounter}\todo[#1]{\thetodocounter: #2}}

\imprimirfolhaderosto

% ----
%Resumo ----

   
\setlength{\absparsep}{18pt} % ajusta o espaçamento dos parágrafos do resumo
\begin{resumo}
  No Brasil, os concursos públicos se configuram como portais para a tão almejada ascensão social, oferecendo estabilidade profissional e remunerações atraentes por um longo período. A alta atratividade desses cargos, por sua vez, intensifica a concorrência, exigindo dos candidatos um alto nível de dedicação e compromisso com os estudos.

    É nesse contexto que surge o Studyflow, uma plataforma inovadora que impulsiona a jornada de quem busca a aprovação em concursos públicos através da tecnologia. O grande diferencial do Studyflow reside na integração de inteligência artificial para a criação de rotinas de estudo personalizadas, moldadas ao ritmo individual de cada estudante e aos requisitos específicos do concurso almejado.

    O Studyflow se mostrou uma ferramenta eficiente e eficaz, com poder de otimizar muito tempo dos usuários, permitindo a maior possibilidade de aprovação em um concurso público.

 \textbf{Palavras-chaves}: Inteligência Artificial, Rotina, Concurso Público, Organização de Estudo.
\end{resumo}

    
    

\cleardoublepage
%-----
\pdfbookmark[0]{Abstract}{abstract} % Sets a PDF bookmark for the abstract

\begin{resumo}[Abstract]
 \begin{otherlanguage*}{english}

     In Brazil, civil service exams are the gateway to the much-desired social ascension, offering professional stability and attractive salaries over a long period of time. The high attractiveness of these positions, in turn, intensifies competition, demanding a high level of dedication and commitment from candidates.


    It is in this context that Studyflow emerges, an innovative platform that boosts the journey of those seeking to pass public exams through technology. Studyflow's great advantage lies in the integration of artificial intelligence to create personalized study routines, tailored to the individual rhythm of each student and the specific requirements of the competition they are aiming for.


    Studyflow has proved to be an efficient and effective tool, with the power to optimize users' time, enabling them to have a greater chance of passing a public examination.
   \noindent 
   
   \textbf{Key-words}: Artificial Intelligence, Routine, Public Tender, Study Organization.
 \end{otherlanguage*}
\end{resumo}

\cleardoublepage

% ---
% inserir lista de ilustrações
% ---
\pdfbookmark[0]{\listfigurename}{lof}
\listoffigures*
\cleardoublepage
% ---

% ---
% inserir lista de tabelas
% ---
\pdfbookmark[0]{\listtablename}{lot}
\listoftables*
\cleardoublepage
% ---

% ---
% inserir lista de quadros
% ---
\pdfbookmark[0]{\listofquadrosname}{loq}
\listofquadros*
\cleardoublepage
% ---

\input{pre-siglas}
%\todo[inline]{Remover lista de simbolos se não for necessário}


% ---
% inserir o sumario
% ---
\pdfbookmark[0]{\contentsname}{toc}
\tableofcontents*
\cleardoublepage
% ---


% ----------------------------------------------------------
% ELEMENTOS TEXTUAIS
% ----------------------------------------------------------
\textual


\input{Std.1 - Introducao}
\chapter{Revisão da Literatura}


A seção de revisão da literatura tem como objetivo apresentar pesquisas, artigos, livros ou afins já realizados por outros autores, que embasem as problematizações utilizadas como motivo para o desenvolvimento da aplicação StudyFlow. Isto é, serão abordadas fontes de informação que comprovam a utilidade e pertinência da aplicação.

\section{Dificuldade em organizar rotina para estudos}

Com o crescimento exponencial da internet, cada vez mais informações são disponibilizadas ao público. No entanto, para aqueles que buscam conteúdo específico para estudar para concursos públicos, muitos dos resultados apresentados em uma busca regular em mecanismos de pesquisa podem ser irrelevantes ou incompletos.

Além da dificuldade em encontrar o conteúdo relevante para os concursos, os estudantes enfrentam desafios adicionais ao tentar organizar todas as informações. Esta é uma das maiores barreiras para aqueles que ainda não adotaram nenhum método ou rotina de estudos. Lia Salgado, autora do livro "Como vencer a maratona dos concursos públicos", destaca: "Esse é o primeiro impacto, mesmo. É assustador. Eu senti isso na pele quando comecei a minha preparação." sobre o volume de conteúdo a ser estudado para concursos. "A solução é organizar o estudo, planejar a rotina diária para ter o momento certo de estudar e distribuir as matérias ao longo da semana."

\section{A importância da Inteligência Artificial no apoio dos estudos}

A \ac{ia} está revolucionando diversos setores da sociedade, e a educação não é exceção. Nos últimos anos, ferramentas e plataformas impulsionadas por IA vêm surgindo com o objetivo de auxiliar os alunos em sua jornada de aprendizado, tornando-a mais personalizada, eficiente e eficaz.


Em destaque, os principais benefícios da \acs{ia} na educação é a sua capacidade de personalizar o ensino de acordo com as necessidades individuais de cada aluno. Através de algoritmos de aprendizado de máquina, os sistemas de \acs{ia} podem analisar o desempenho, estilo de aprendizado e ritmo de cada estudante, adaptando o conteúdo, as atividades e os métodos de ensino de forma otimizada \cite{shemshack2020systematic}.

A utilização de ferramentas impulsionadas por \ac{ia} não é novidade no mercado, mas a partir de 2022 houve um boom nesse mercado com a chegada do  ChatGPT, criado pela OpenAI. Para fins estudantis, os chatbots são as ferramentas amplamente adotadas, por sua facilidade de desenvoltura e compreensão pelo usuário.
Os Chatbots podem responder a dúvidas sobre a matéria, sistemas de tutoria oferecem exercícios personalizados e plataformas de aprendizado adaptativo sugerem conteúdos relevantes para cada estudante \cite{silva2023inteligencia}. Além disso, ferramentas de reconhecimento de fala e texto podem auxiliar alunos com deficiências e softwares de tradução podem facilitar o aprendizado em diferentes idiomas.





\chapter{GESTÃO DO PROJETO}
Nesta seção, serão apresentados os métodos escolhidos para a gestão do projeto e da equipe, com o objetivo de assegurar a melhor utilização possível do tempo, orçamento e recursos voltados para o projeto, para que esse possa ser concluído dentro do prazo estabelecido. Também serão levantados alguns riscos possíveis, afim de que com o conhecimento dessas possibilidades, medidas possam ser tomadas para evitá-los.  

\section{Formação da equipe}
A equipe foi formalizada durante as aulas da disciplina, porém já havia sido definida posteriormente. Todos os integrantes são alunos do curso de Tecnologia em Análise e Desenvolvimento de Sistemas, do Instituto Federal de São Paulo (IFSP), campus São Paulo. A equipe se baseia nos conhecimentos de cada um de seus membros com o objetivo de preencher as necessidades do projeto. Os participantes da equipe Noz são: 

\begin{itemize}
    \item \textbf{Kevin Klein}
    \item \textbf{Leonardo Tumani Teixeira Meireles}
    \item \textbf{Luiz Fernando Cavalcante de Faria}
    \item \textbf{Pedro Felipe da Silva Dias}
    \item \textbf{Ruan de Souza Cardoso Brito}
\end{itemize}

\subsection{Papéis}
Os papéis foram definidos através das habilidades dos membros , para que cada um pudesse atuar de maneira segura com seus conhecimentos, dessa maneira a organização e a fluidez do projeto são beneficiadas. As atividades são planejadas para que todos sejam responsáveis por alguma parte específica do projeto, podendo receber ajuda dos outros participantes caso seja necessário.

\newpage
\subsection{Organização da atividades}
    \item \begin{figure}[!htb]
 	    \centering
 	    \caption{\label{logo}Atividades de desenvolvimento}
 	    \includegraphics[width=16cm]{img/tabela-papeis-1.png}
\end{figure}


    \item \begin{figure}[!htb]
 	    \centering
 	    \caption{\label{logo}Atividades de gestão e planejamento}
 	    \includegraphics[width=16cm]{img/tabela-papeis-2.png}
\end{figure}

\section{Gestão de tempo e desenvolvimento}
A equipe decidiu aderir à utilização do Scrum como framework de gerenciamento, a fim de melhorar a organização durante o desenvolvimento do projeto. Ele foi escolhido pela sua eficiência e ampla utilização por diversas empresas no mercado, além de ser conhecido pelos integrantes do grupo, o que facilita sua implementação.  Além disso também optamos pela utilização do Kanban, para garantir a eficiência na realização das tarefas e o cumprimento dos prazos.

\subsection{Scrum}
O Scrum é uma metodologia de desenvolvimento ágil amplamente empregada para lidar com a complexidade na criação de produtos. Este método valoriza a colaboração, a autonomia da equipe e a entrega progressiva e iterativa. Composto por uma série de práticas, papéis e artefatos, o Scrum promove a eficácia e a qualidade do trabalho realizado, impulsionando a entrega de valor de forma consistente ao longo do tempo.

\subsection{Kanban}
O Kanban é uma metodologia de gestão visual que teve origem no Japão e ganhou popularidade em diversos setores, como desenvolvimento de software, manufatura e serviços. O termo "Kanban" significa "sinal visual" em japonês, e essa abordagem se baseia na utilização de cartões ou post-its para representar unidades de trabalho e visualizar o fluxo do processo. Essa metodologia visa proporcionar transparência sobre o trabalho em andamento e controlar o trabalho em progresso (WIP) para otimizar a eficiência do sistema.

\section{Gestão de comunicação}
A comunicação é parte essencial para que tudo corra bem no projeto. Foram utilizados alguns meios para realizar esse diálogo entre a equipe.

O meios de comunicação utilizados internamente foram o  Whatsapp, aplicativo de mensagens instantâneas  e chamadas de voz, foi usado para troca de mensagens durantes as semanas, a fim de proporcionar agilidade e facilidade na comunicação, e o Discord, que é uma aplicação voltada para a comunicação, principalmente de grupos e comunidades, foi usado para as reuniões realizadas semanalmente.

Para a comunicação com o público foram criados um blog, na plataforma Blogger, onde são compartilhadas as atualizações semanais e informações relevantes sobre o projeto.

 \begin{figure}[!htb]
 	    \centering
 	    \caption{\label{logo}QRCode do Blog}
 	    \includegraphics[width=5cm]{img/qrcode-blog.png}
 	    \fonte{Os autores.}
\end{figure}
\FloatBarrier

 \begin{figure}[!htb]
 	    \centering
 	    \caption{\label{logo}QRCode do Youtube}
 	    \includegraphics[width=5cm]{img/qrcode-youtube.png}
 	    \fonte{Os autores.}
\end{figure}
\FloatBarrier

\section{Análise de riscos}
Nessa seção é possível avaliar alguns dos possíveis riscos ao projeto, analisando seu nível de impacto e qual o tipo de resposta para cada um em específico.

\begin{figure}[!htb]
 	    \centering
 	    \caption{\label{logo}Análise de riscos}
 	    \includegraphics[width=15cm]{img/riscos.png}
\end{figure}

Segue abaixo uma breve explicação sobre cada um dos possíveis riscos ao projeto:

\begin{itemize}
    \item \textbf{Desistência pessoal:} Membros da equipe abandonam o projeto, causando lacunas na expertise e sobrecarregando os membros restantes.
    \item \textbf{Problemas de saúde:} Membros da equipe enfrentam problemas de saúde que afetam sua capacidade de contribuir para o projeto.
    \item \textbf{Conflitos interpessoais:} Desentendimentos ou tensões entre membros da equipe prejudicam a colaboração e a eficiência do projeto.
    \item \textbf{Comprometimento com outras tarefas:} Membros da equipe têm prioridades divididas entre várias tarefas ou projetos, resultando em atrasos ou falta de dedicação ao projeto em questão.
    \item \textbf{Mudança de requisitos:} Alterações nos requisitos do projeto após o início do desenvolvimento, levando a retrabalho e atrasos.
    \item \textbf{Falhas na comunicação:} Comunicação inadequada entre membros da equipe, clientes ou partes interessadas, levando a mal-entendidos e erros.
    \item \textbf{Falta de conhecimento:} Membros da equipe não possuem as habilidades ou conhecimentos necessários para concluir com sucesso determinadas tarefas ou aspectos do projeto.
    \item \textbf{Instabilidade na rede:} Problemas com a conexão de rede afetam a colaboração remota ou o acesso a recursos necessários para o projeto.
    \item \textbf{Falhas de hardware:} Hardware essencial para o projeto falha, causando interrupções no desenvolvimento ou perda de dados.
    \item \textbf{Escopo mal definido:} Requisitos do projeto não estão claramente definidos desde o início, levando a confusão e revisões frequentes.
    \item \textbf{Desempenho insatisfatório:} O produto final não atende às expectativas de desempenho dos usuários, levando à insatisfação e possível rejeição.
    \item \textbf{Falha de segurança:} Vulnerabilidades de segurança no sistema comprometem a integridade ou a confidencialidade dos dados, resultando em riscos para os usuários e para a empresa.
    \item \textbf{Falha em tecnologias externas:} Dependência de tecnologias externas que podem falhar ou não atender às expectativas, afetando o desenvolvimento ou o funcionamento da aplicação.
    \item \textbf{Problema no treinamento da IA:} Dificuldades no treinamento de sistemas de inteligência artificial para alcançar os resultados desejados, resultando em desempenho inadequado ou inexato.
\end{itemize}
\usepackage{booktabs}
\usepackage{caption}
\usepackage{array}

\chapter{DESENVOLVIMENTO DO PROJETO}

Este tópico apresenta informações do produto, desenvolvimento do \textit{software},
tempo para execução, ferramentas utilizadas, viabilidade e riscos.

\section{Métricas}

\begin{table}[!htp]
\centering
\caption{Métricas do projeto}\label{tab: }

\begin{tabular}{l {c} {c} {c} {c} {c} {c} {c} {c}}
\multicolumn{2}{c}{Itens} & \multicolumn{6}{c}{Progresso do Projeto} \\
\cline{2-8}
 & 12/4 & 19/4 & 26/4 & 03/5 & 10/5 & 17/5 & 26/7 \\
\hline
Reuniões & 1 & 2 & 3 & 4 & 5 & 6 & 9 \\
Posts de Blog & 5 & 5 & 6 & 7 & 7 & 7 & 7\\
Vídeos & 1 & 2 & 2 & 2 & 3 & 3 & 3\\
Requisitos & 12 & 12 & 19 & 19 & 19 & 19 & 19\\
Entidades de BD & 0 & 0 & 6 & 6 & 6 & 6 & 6\\
Commits & 3 & 4 & 6 & 9 & 23 & 25 & 30\\
Testes Unitários Quantidade & 0 & 0 & 0 & 0 & 0 & 0 & 0\\
\end{tabular}
\end{table}


\section{Arquitetura da solução}
O projeto web é dividido em duas camadas principais, sendo elas o \textit{front-end}, responsável pela estilização da plataforma e interação com o usuário, e o \textit{back-end}, responsável pela aplicação das regras de negócio, gestão das informações em um banco de dados, e pela lógica de execução em si da plataforma.

 \begin{figure}[!htb]
 	    \centering
 	    \caption{\label{logo}Arquitetura}
 	    \includegraphics[width=15cm]{img/infra-model.png}
 	    \fonte{Os autores.}
\end{figure}

\subsection{Front-end}
O Front-end é a interface da aplicação, construída com Next.js, um \textit{framework} React que oferece otimizações de desempenho e server-side rendering para uma experiência de usuário mais rápida e fluida. No desenvolvimento da interface, são empregadas as seguintes tecnologias:

\begin{itemize}
    \item \textbf{Tailwind CSS} \newline
Tailwind CSS é uma ferramenta utilizada para a estilização da aplicação. Ele adota uma abordagem de \textit{utility-first}, o que significa que as classes \ac{css} são utilizadas diretamente no \ac{html} para estilizar os elementos. Isso proporciona uma experiência de desenvolvimento mais rápida e consistente, além de facilitar a manutenção do código.

    \item \textbf{ShadCN} \newline
ShadCN é uma coleção de componentes prontos que podem ser importados e customizados dentro do código. Esses componentes são escritos em \textit{Typescript} e Tailwind \acs{css}. Ele não é considerado uma biblioteca, já que é uma extensão do Radix, outra biblioteca de estilização para Javascript.

    \item \textbf{Moment} \newline
Moment.js é uma biblioteca popular para manipulação de datas e horas em JavaScript. Ela oferece uma ampla gama de funcionalidades para formatação, análise e manipulação de datas, tornando mais fácil trabalhar com informações temporais na aplicação.

    \item \textbf{Nodemailer} \newline
Nodemailer é uma biblioteca utilizada para enviar e-mails através de Node.js. Ela oferece uma interface simples e flexível para o envio de \textit{e-mails}, permitindo configurar facilmente o servidor de e-mail, criar templates personalizados e enviar mensagens de forma assíncrona.

    \item \textbf{Framer Motion} \newline
Framer Motion é uma biblioteca de animações para React que facilita a criação de animações fluidas e responsivas em componentes da interface. Ela oferece uma \ac{api} declarativa e intuitiva para definir animações de entrada, saída e transição, além de suportar gestos e interações do usuário.

    \item \textbf{PDF Viewer} \newline
PDF Viewer é uma biblioteca Javascript projetada especificamente para a leitura de arquivos em formato \acs{pdf} enviados pelos usuários, dentro do NodeJS. Com uma série de ferramentas avançadas, oferece uma experiência de visualização personalizada e intuitiva desses documentos.

    \item \textbf{Open AI} \newline
A biblioteca OpenAI permite que os desenvolvedores usem os modelos de \acs{ia} generativos de texto desenvolvidos pela OpenAI, como o GPT 3.5, para criar recursos e aplicativos com tecnologia de \acs{ia}.

    \item \textbf{Bibliotecas do Firebase} \newline
Dentro do \textit{Front-end}, são utilizadas diversas bibliotecas do Firebase para interação com o Back-end e execução de funcionalidades como autenticação, armazenamento de dados e comunicação em tempo real. Algumas das bibliotecas comumente utilizadas incluem:
\begin{itemize}
    \item \textbf{firebase}
    \item \textbf{firebase-admin}
    \item \textbf{firebase-functions}
    \item \textbf{firebase-tools}
\end{itemize}

Essas bibliotecas fornecem uma integração simplificada entre o Front-end e o Back-end, permitindo o desenvolvimento de uma aplicação robusta e interativa.

\end{itemize}

\subsection{Back-end}
Para o Back-end, é utilizado Firebase, que fornece serviços de banco de dados, armazenamento, autenticação e hospedagem, entre outros, de forma simplificada. O Firebase permite uma configuração rápida e fácil, facilitando o desenvolvimento e a viabilização do projeto.
Dentro da plataforma do Firebase, são utilizadas as seguintes funcionalidades:

\begin{itemize}
\item \textbf{Firebase Auth} \newline
Para autenticação de usuários, permitindo login com e-mail, redes sociais, entre outros métodos.
\item \textbf{Firebase Firestore} \newline
Para armazenamento e gerenciamento de dados em tempo real, oferecendo um banco de dados NoSQL escalável e altamente disponível.
\item \textbf{Firebase Storage} \newline
Para armazenamento de arquivos, como imagens e vídeos, diretamente na infraestrutura do Firebase.
\end{itemize}

\subsection{Banco de dados}
O banco de dados da aplicação está contido nos serviços oferecidos pelo Firestore, que é um banco de dados NoSQL escalável e altamente disponível. A comunicação entre as camadas, \acs{api} externas e com o cliente são realizadas através do Protocolo \ac{http} e chamadas \ac{rest}.
O banco de dados é estruturado da seguinte maneira, a fim de suportar a gestão das informações dentro da plataforma:

\begin{itemize}
    \item \textbf{Tabela users}
    
    \begin{itemize}
    \item \textbf{fullname:} Armazena o nome completo do usuário.
    \item \textbf{email:} Guarda o endereço de e-mail do usuário.
    \item \textbf{document:} Pode armazenar o documento de identificação do usuário, como CPF. 
    \item \textbf{noticeid:} Chave estrangeira que faz referência ao edital (notice) associado ao usuário.
    \end{itemize}
    
    \item \textbf{Tabela tasks}
    
    \begin{itemize}
    \item \textbf{title:} Título da tarefa.
    \item \textbf{description:} Descrição detalhada da tarefa.
    \item \textbf{contentid:} Chave estrangeira que referencia o conteúdo (content) associado à tarefa. 
    \item \textbf{difficultyid:} Chave estrangeira que referencia o nível de dificuldade da tarefa. 
    \item \textbf{hasfinished:} Indica se a tarefa foi concluída ou não.
    \item \textbf{userid:} Chave estrangeira que faz referência ao usuário que criou a tarefa. 
    \item \textbf{dayofweek:} Dia da semana em que a tarefa deve ser realizada.
    \item \textbf{startat:} Horário de início da tarefa.
    \item \textbf{finishat:} Horário de término da tarefa.
    \end{itemize}
    
    \item \textbf{Tabela difficulties}
    
    \begin{itemize}
    \item \textbf{name:} Nome do nível de dificuldade.
    \item \textbf{displayname:} Nome de exibição do nível de dificuldade.
    \end{itemize}
    
    \item \textbf{Tabela subjects (matérias)}
    \begin{itemize}
    \item \textbf{name:} Nome da matéria.
    \item \textbf{noticeid:} Chave estrangeira que faz referência ao edital (notice) associado à matéria.
    \end{itemize}
    
    \item \textbf{Tabela contents}
    \begin{itemize}
    \item \textbf{subjectid:} Chave estrangeira que referencia a matéria (subject) associada ao conteúdo. 
    \item \textbf{text:} Texto do conteúdo, que pode conter informações relevantes para estudo.
    \end{itemize}
    
    \item \textbf{Tabela notices (editais)}
    \begin{itemize}
    \item \textbf{name:} Nome do cargo ou edital.
    \item \textbf{filesrc:} Caminho para o arquivo do edital. 
    \item \textbf{userid:} Usuário que fez o upload do edita.
    \end{itemize}
    
\end{itemize}

Este modelo de banco de dados é projetado para permitir a associação de usuários a tarefas específicas, associadas a conteúdos de estudo e matérias específicas relacionadas aos editais. A inclusão de um nível de dificuldade (na tabela difficulties) proporciona uma maneira de classificar a complexidade das tarefas, enquanto a tabela notices permite o armazenamento e acesso aos editais relacionados aos estudos.

 \begin{figure}[!htb]
 	    \centering
 	    \caption{\label{logo}Modelo de classes do banco de dados}
 	    \includegraphics[width=15cm]{img/db-model.png}
 	    \fonte{Os autores.}
\end{figure}
\FloatBarrier

\subsection{Integrações}
Para a interpretação dos conteúdos pragmáticos dentro dos editais, faremos uso dos modelos de inteligência artificial fornecidos pela OpenAI. A OpenAI é uma organização de pesquisa em \acs{ia} conhecida por seus avançados modelos de \acs{ia}, especialmente na área de  \ac{nlp}. Iremos utilizar esses modelos para desenvolver os seguintes motores:

\begin{itemize}
\item \textbf{Interpretação de Conteúdo de Edital e Geração de Matérias} \newline
Este motor é encarregado de interpretar o conteúdo pragmático após a filtragem do edital fornecido pelo usuário. Ele irá identificar e extrair as matérias que serão cobradas no concurso, populando assim o banco de dados.

\item \textbf{Interpretação de Matérias e Geração de Tarefas e Rotinas de Estudo} \newline
 Com o banco de dados já contendo as matérias identificadas, este motor entra em ação para gerar tarefas e rotinas de estudo personalizadas. Recebendo como entrada as matérias e conteúdos que o usuário ainda precisa estudar, ele irá gerar um cronograma de estudo detalhado, distribuindo as tarefas ao longo da semana de acordo com as necessidades e preferências do usuário.
\end{itemize}

Essas integrações permitem uma abordagem mais eficiente e personalizada no processo de estudo para concursos, aproveitando o poder dos modelos de linguagem avançados fornecidos pela OpenAI.

\subsection{Versionamento de Código}
O versionamento de código é uma prática fundamental no desenvolvimento de software, permitindo o controle e gerenciamento das alterações feitas ao longo do tempo em um projeto. Para isso, utilizaremos o \textit{GitHub} como plataforma de versionamento, que oferece uma série de recursos poderosos para colaboração e controle de versões. No nosso ambiente de desenvolvimento, teremos um repositório principal hospedado no GitHub:

\textbf{Repositório do \textit{Front-end} e \textit{Functions} do Firebase:} \newline
Este repositório conterá o código-fonte do \textit{Front-end} desenvolvido com Next.js, bem como as \textit{Functions} do Firebase utilizadas no Back-end. Será organizado de forma a separar claramente os diretórios relacionados ao Front-end e às functions do Firebase, mantendo uma estrutura de pastas intuitiva e coesa.

\textbf{Estratégia de Versionamento: Gitflow} \newline
Para gerenciar as diferentes etapas de desenvolvimento e garantir uma colaboração eficiente entre os membros da equipe, adotaremos a estratégia de versionamento Gitflow. Essa abordagem define um modelo de fluxo de trabalho baseado em branches, que facilita a organização das funcionalidades em desenvolvimento, testes e produção.
Principais \textit{Branches}:

\begin{itemize}
\item \textbf{Main:} Esta \textit{branch} representa a versão estável e de produção do código. Todo o código que está pronto para ser implantado em ambiente de produção é mesclado nesta \textit{branch}.

\item \textbf{Develop:} Esta \textit{branch} é onde o desenvolvimento ativo ocorre. É a \textit{branch} de integração para novas funcionalidades e correções de bugs. Todo o desenvolvimento é feito a partir desta \textit{branch}.

\item \textbf{Feature Branches:} Para cada nova funcionalidade ou tarefa, uma nova \textit{branch} de \textit{feature} é criada a partir da \textit{branch develop}. Esta \textit{branch} é utilizada para implementar a funcionalidade de forma isolada, antes de ser integrada de volta à \textit{branch develop}.
\end{itemize}
 
Adotando essa estratégia de versionamento com o Gitflow, garantimos um desenvolvimento organizado, facilitando a colaboração entre os membros da equipe e mantendo um histórico claro e estruturado das alterações feitas no código-fonte ao longo do tempo.

\subsection{Infraestrutura}
Para hospedagem, optaremos por utilizar os serviços especializados de hospedagem da Vercel para o \textit{Front-end} e do Firebase para o \textit{Back-end} customizado.

\textbf{Hospedagem do Front-end} \newline
A Vercel oferece um serviço de hospedagem altamente escalável e otimizado para aplicações \textit{Front-end}, como o nosso desenvolvido com Next.js. Utilizando a plataforma da Vercel, podemos implantar e hospedar facilmente nosso Front-end, garantindo uma experiência de usuário rápida e confiável.

\textbf{Hospedagem do Back-end no Firebase} \newline
O Firebase oferece por padrão a hospedagem de seus serviços diretamente em sua plataforma, eliminando a necessidade de recorrer a soluções terceirizadas para essa finalidade. Essa integração nativa proporciona uma infraestrutura completa e integrada, capaz de suportar todas as necessidades de nossa aplicação de forma eficiente e escalável.

\subsection{Escalabilidade}
Tanto a Vercel quanto o Firebase oferecem opções de escalabilidade conforme as necessidades do projeto. No caso da Vercel, podemos facilmente escalar nossa aplicação Front-end de acordo com o aumento da demanda de tráfego. Já o Firebase, além de oferecer hospedagem escalável, também permite dimensionar automaticamente o banco de dados e outros serviços conforme necessário.

\subsection{Convenções e Padronização de Código}

Convenções são acordos ou regras estabelecidas para padronizar a forma como realizamos determinadas atividades ou interações. No contexto do desenvolvimento de software, as convenções de codificação são diretrizes estabelecidas para padronizar a escrita e a organização do código-fonte de uma aplicação. Elas definem como devemos nomear variáveis, formatar o código, documentar funcionalidades e adotar certas práticas de desenvolvimento.

Para esse projeto, iremos seguir com as seguintes convenções e padrões de código:

\begin{itemize}
    \item Nomenclatura de Variáveis e Funções:

Utilize nomes descritivos e significativos para variáveis e funções.
Prefira \textit{camelCase} para nomes de variáveis e funções em JavaScript/TypeScript.

    \item Comentários:
Inclua comentários claros e concisos para explicar trechos de código complexos ou de difícil compreensão.
Evite comentários óbvios que apenas repetem o que o código faz.

    \item Indentação e Formatação:

Utilize uma tabulação consistente para indentação, preferencialmente com 2 ou 4 espaços.
Mantenha linhas de código com até 80-100 caracteres para facilitar a leitura em telas menores.
Organize o código de forma clara e coesa, utilizando espaços em branco para separar blocos lógicos.

    \item Tratamento de Erros:

Sempre inclua tratamento de erros adequado em pontos críticos do código.
Utilize \textit{try-catch} para capturar e lidar com exceções de forma apropriada.

    \item Gerenciamento de Dependências:

Mantenha uma lista atualizada de todas as dependências e suas versões no arquivo de manifesto (como package.json).
Utilize um gerenciador de dependências confiável, como \acs{npm} ou \acs{yarn}, e evite adicionar dependências desnecessárias.

    \item Revisões de Código:

Realize revisões de código regulares entre os membros da equipe para identificar e corrigir problemas de qualidade, estilo e desempenho.
Mantenha um ambiente colaborativo e aberto para sugestões e melhorias no código.

\end{itemize}

\newpage


\input{Std.4.2 - Historias do usuário}
\section{Fases de Entrega}

As fases de entrega de um projeto são etapas fundamentais que conduzem o desenvolvimento de uma solução desde sua concepção até sua implementação e lançamento. Cada fase representa um marco importante no ciclo de vida do projeto, onde objetivos específicos são alcançados e progresso significativo é realizado.

Nossas fases de entrega planejadas são:

\begin{itemize}
\item Planejamento e Análise:
Nesta fase, são identificados os requisitos do projeto, definidos os objetivos e escopo, e elaborado o plano de projeto detalhado. Também é realizada uma análise de viabilidade técnica e financeira.

\item Prova de Conceito (PoC):
A PoC é uma fase inicial do projeto na qual são desenvolvidos protótipos ou demonstrações que validam a viabilidade técnica das principais funcionalidades da plataforma. Nesta fase, focamos em implementar um conjunto mínimo de recursos para validar a solução proposta e demonstrar sua viabilidade.

\item Desenvolvimento Iterativo:
Após o sucesso da PoC, o desenvolvimento da plataforma começa em etapas iterativas. Funcionalidades adicionais são implementadas em ciclos de desenvolvimento curtos, permitindo feedback contínuo e ajustes conforme necessário.

\item Testes e Qualidade:
Durante todo o processo de desenvolvimento, são realizados testes rigorosos para garantir que a plataforma atenda aos requisitos de qualidade e segurança. Isso inclui testes de unidade, integração, aceitação do usuário e segurança.

\item Implantação e Lançamento:
Após a conclusão do desenvolvimento e dos testes, a plataforma é implantada em um ambiente de produção e está pronta para ser lançada. Isso pode incluir a configuração de servidores, migração de dados e treinamento de usuários.

\item Monitoramento e Manutenção:
Após o lançamento, a plataforma é continuamente monitorada para garantir que esteja funcionando conforme o esperado. Também são feitas atualizações regulares e manutenção para corrigir bugs, adicionar novos recursos e melhorar a experiência do usuário.

\end{itemize}

\textbf{Prova de Conceito (PoC) Inicial:}
Na PoC inicial, nosso foco seria desenvolver um protótipo funcional da plataforma que demonstre as principais funcionalidades propostas. Isso pode incluir a autenticação de usuários, a criação de tarefas e a integração com serviços de terceiros, como o Firebase para o back-end e a Vercel para o front-end. A PoC permite validar a viabilidade técnica da solução proposta e identificar possíveis desafios ou obstáculos que precisam ser superados antes da implementação completa.

\textbf{Entrega Final da Plataforma:}
Na entrega final da plataforma, todos os recursos e funcionalidades planejados são implementados e testados completamente. Isso inclui a implementação completa e treinamento dos modelos de IA, adequação a políticas de segurança, a integração com serviços de terceiros, como autenticação, armazenamento de dados e hospedagem, e a garantia de que a plataforma atenda aos requisitos de desempenho, escalabilidade e usabilidade. A entrega final marca o lançamento oficial da plataforma e seu uso pelos usuários finais.
\section{Segurança}

A segurança da aplicação é uma preocupação primordial em qualquer sistema de software, especialmente quando se trata de lidar com dados sensíveis dos usuários. A proteção dessas informações é fundamental não apenas para garantir a confiança dos usuários, mas também para cumprir regulamentações de privacidade e evitar violações de dados.

Os dados dos usuários, como informações pessoais, registros de atividades e preferências, são ativos valiosos que precisam ser protegidos contra ameaças internas e externas. Isso se torna ainda mais crucial em um cenário onde a coleta, armazenamento e processamento de dados são cada vez mais presentes no mundo digital.

Além disso, as regulamentações de privacidade, como a \ac{lgpd}, estabelecem diretrizes claras sobre como as organizações devem lidar com os dados pessoais dos usuários, impondo penalidades rigorosas para aqueles que não cumprem os requisitos de segurança e privacidade.

Portanto, é essencial que adotemos medidas proativas para garantir a segurança dos dados de seus usuários. Isso inclui a implementação de práticas de segurança robustas em todos os níveis da aplicação, desde o desenvolvimento seguro de código até a proteção dos dados em repouso e em trânsito, bem como o controle de acesso apropriado aos recursos do sistema.

Ao priorizar a segurança da aplicação, não apenas estamos protegendo os dados confidenciais dos usuários, mas também estamos demonstrando nosso compromisso com a privacidade e a confiabilidade do serviço que oferecemos. Isso fortalece a relação de confiança com os usuários e contribui para o sucesso a longo prazo da aplicação.

\subsection{Autenticação e Autorização}

A autenticação e autorização são pilares fundamentais da segurança da aplicação, desempenhando papéis essenciais na proteção dos dados e na prevenção de acessos não autorizados. Vamos explorar como esses conceitos são implementados para garantir a segurança da nossa plataforma.

\textbf{Autenticação:}
A autenticação é o processo pelo qual a identidade de um usuário é verificada, garantindo que apenas usuários legítimos tenham acesso aos recursos da aplicação. Para isso, adotaremos um sistema robusto de autenticação baseado em \ac{jwt}. Quando um usuário faz login na plataforma, suas credenciais são verificadas e, se válidas, um token \acs{jwt} é gerado e enviado de volta ao cliente. Esse token contém informações sobre a identidade do usuário e é assinado digitalmente para garantir sua autenticidade.

O Firebase desempenha um papel crucial na interação com os tokens \acs{jwt}, facilitando a autenticação segura dos usuários em nossa aplicação. Aqui está um resumo de como o Firebase interage com os tokens:

\begin{itemize}
\item \textbf{Geração de Tokens JWT:}
Quando um usuário realiza o processo de autenticação no Firebase, seja por e-mail e senha, login social (como \textit{Google} ou \textit{Facebook}) ou outros métodos suportados, o Firebase gera um token \acs{jwt} válido que contém informações sobre a identidade do usuário, como seu \acs{id} exclusivo, endereço de e-mail e quaisquer metadados adicionais necessários para autorização.

\item \textbf{Validação dos Tokens:}
O Firebase verifica a validade dos tokens \acs{jwt} antes de conceder acesso aos recursos protegidos. Isso inclui a verificação da assinatura digital do token para garantir sua autenticidade e a verificação de outros atributos, como a data de expiração, para garantir que o token ainda esteja válido.

\item \textbf{Decodificação dos Tokens:}
O Firebase decodifica os tokens \acs{jwt} para extrair as informações relevantes sobre o usuário. Isso pode incluir o \acs{id} do usuário, endereço de e-mail e quaisquer outros dados personalizados incluídos no token durante a geração.

\item \textbf{Gerenciamento de Sessões:}
O Firebase gerencia automaticamente as sessões de autenticação dos usuários, mantendo os tokens \acs{jwt} válidos e atualizados durante o período de uso da aplicação. Isso permite que os usuários permaneçam autenticados em diferentes dispositivos e sessões sem a necessidade de reautenticação frequente.

\item \textbf{Controle de Acesso:}
Com base nas informações contidas nos \textit{tokens} \acs{jwt}, o Firebase aplica políticas de autorização para determinar quais recursos e operações o usuário autenticado tem permissão para acessar. Isso inclui a aplicação de regras de segurança personalizadas em serviços como o \textit{Firebase Realtime Database} e o \textit{Cloud} Firestore.

\end{itemize}

Em resumo, o Firebase oferece uma integração completa e segura com \textit{tokens} \acs{jwt}, facilitando a autenticação e autorização dos usuários em nossa aplicação. Essa abordagem robusta garante a proteção dos dados dos usuários e mantém a segurança da aplicação em todos os níveis.

\textbf{Autorização:}
A autorização, por sua vez, determina quais recursos e operações um usuário autenticado tem permissão para acessar e executar. Para isso, implementaremos controles de acesso granulares em toda a plataforma. Cada solicitação feita por um usuário será verificada em relação às políticas de autorização configuradas. Isso inclui restrições de acesso a determinadas funcionalidades com base no papel do usuário, como administrador, moderador ou usuário padrão.

Ao implementar um sistema de autenticação e autorização sólido, estamos fortalecendo as defesas da nossa aplicação contra ameaças de segurança. O uso de \textit{tokens} \acs{jwt} oferece uma camada adicional de segurança, pois eles são assinados digitalmente e podem incluir informações sobre o usuário, como permissões e papéis, que são verificados em cada solicitação.

Além disso, ao adotar práticas de segurança recomendadas, como criptografia de dados, gerenciamento seguro de credenciais e monitoramento de atividades suspeitas, estamos reduzindo significativamente o risco de violações de segurança e protegendo os dados sensíveis dos nossos usuários.

\subsection{Políticas de Segurança}

As políticas de segurança desempenham um papel vital na proteção dos ativos e dados de uma plataforma contra ameaças cibernéticas. Elas estabelecem diretrizes e regras para garantir a confidencialidade, integridade e disponibilidade dos dados, bem como para prevenir acessos não autorizados e mitigar riscos de segurança.

No contexto da nossa plataforma, as políticas de segurança são aplicadas em várias áreas. Implementamos políticas de controle de acesso e identidade para garantir que apenas usuários autorizados tenham acesso aos recursos e funcionalidades da plataforma. Isso inclui a utilização de autenticação forte, controle de acesso baseado em função e monitoramento de atividades suspeitas.

Além disso, estabelecemos políticas de criptografia para proteger os dados sensíveis dos usuários durante o armazenamento e a transmissão. Os dados são criptografados de forma a garantir que somente as partes autorizadas tenham acesso às informações, mesmo que ocorra uma violação de segurança.

Outro aspecto importante é a implementação de políticas de monitoramento e resposta a incidentes. Realizamos monitoramento contínuo dos sistemas e registros de atividades para detectar e responder rapidamente a possíveis ameaças. Isso inclui a realização de auditorias de segurança regulares e a elaboração de planos de resposta a incidentes para lidar com situações de emergência.

Em resumo, as políticas de segurança são essenciais para proteger os dados dos usuários e garantir a confiabilidade e integridade da nossa plataforma. Ao implementar políticas de segurança robustas e alinhadas às melhores práticas do setor, estamos comprometidos em oferecer um ambiente seguro e confiável para nossos usuários.
%\input{Std.4.5 - Identidade Visual}
%\input{Std.4.6 - Plano de Testes}
\section{Viabilidade Financeira}

Inicialmente, para a fase de desenvolvimento e testes, vamos utilizar os planos gratuitos oferecidos pela Vercel para hospedar nosso \textit{front-end}. Essa escolha nos permitirá iniciar o projeto de maneira ágil e econômica, facilitando a implantação e os testes iniciais da aplicação.

No entanto, para a fase de produção e lançamento oficial da plataforma, planejamos migrar para soluções pagas. Na Vercel, iremos adotar, inicialmente, o plano Pro, que oferece recursos adicionais, como escalabilidade aprimorada, limites maiores de \textit{bandwidth} e de cachê de dados, além de suporte especializado. Isso garantirá um desempenho consistente e confiável da nossa aplicação em ambiente de produção, além de proporcionar um nível mais alto de serviço e suporte.

Já no Firebase, desde a fase de desenvolvimento iremos optar pelo plano pago Blaze. Isso se dá pois esse plano nos da acesso a serviços específicos da plataforma, como as Functions, funcionalidade que será essencial para o funcionamento da plataforma. Apesar de ser um plano pago, nossos custos iniciais serão baixos, pois o plano segue o formato \textit{"pay-as-you-go"}, ou seja, o faturamento da conta será conforme o uso da plataforma. A flexibilidade desse modelo de pagamento nos permitirá começar com custos mínimos e aumentar conforme o crescimento e a demanda da aplicação.


Essas abordagens nos possibilitam começar com investimentos mínimos durante a fase de desenvolvimento, ajustando nossos gastos de acordo com o crescimento e a maturidade do projeto. Dessa forma, garantimos uma transição suave para o ambiente de produção, maximizando o valor entregue aos usuários finais e garantindo o sucesso contínuo da nossa aplicação.

\subsection{Custos}

Até a criação da aplicação não haverá custos com as tecnologias já citadas. Para a próxima fase da aplicação, o teste de mercado, os custos de se manter a aplicação de pé e até 10 mil usuários foram levantados nas tabelas \ref{tab: Custos Fixos} e \ref{tab: Custos Variaveis}, que tratam dos custos fixos e variaveis respectivamente.

\begin{table}[!htp]
\centering
\caption{Custos Fixos}\label{tab: Custos Fixos}
\scriptsize
\begin{tabular}{lrr}\toprule
Tecnologias &Custo Mensal \\\cmidrule{1-2}
Firestore &15.00 \\\cmidrule{1-2}
Cloud Storage &10.00 \\\cmidrule{1-2}
Cloud Functions &30.00 \\\cmidrule{1-2}
Vercel &100.00 \\\cmidrule{1-2}
Serviço de e-mail &40.00 \\\cmidrule{1-2}
Domínio &45.00 \\\midrule
\bottomrule
\end{tabular}
\fonte{Os autores.}
\end{table}

Os custos variaveis são todos referentes a \acs{api} da OPenAI. Para cada modelo utilizado o custo de requisição se difere. 

\begin{table}[!htp]\centering
\caption{Custos Variaveis}\label{tab: Custos Variaveis}
\scriptsize
\begin{tabular}{lrrrr}\toprule
Tasks IA &Custo por requisição &Requisições mensais &Total Mês \\\cmidrule{1-4}
Leitura de Edital &0.05 &10000 &500.00 \\\cmidrule{1-4}
Geração de Tarefas &0.02 &10000 &200.00 \\\midrule
\bottomrule
\end{tabular}
\end{table}

Ao total, com os custos fixos e váriaveis, para a plataforma do Studyflow se manter no ar e funcionando, com a capacidade de até 10 mil usuários, é necessário a quantia de R\$ 940,00 mensais.
\subsection{Planos de Assinatura e Expectativa Financeira}

Após a análise dos custos envolvidos na operação da plataforma, reconhecemos a necessidade de implementar um modelo de assinatura para garantir o acesso completo às ferramentas oferecidas pela nossa aplicação. Este modelo de assinatura será oferecido em formatos mensal e anual, proporcionando flexibilidade aos usuários de acordo com suas preferências e necessidades. Abaixo, detalhamos os planos disponíveis:

\textbf{Plano Gratuito}

Com o plano gratuito, o usuário terá acesso a todas as funcionalidades da plataforma. Porém, ele está limitado a fazer o upload de apenas um edital, e de utilizar a geração de tarefas por até 3 semanas. Dessa forma, o usuário pode ter uma experiência dentro da plataforma, e decidir se faz sentido ou não começar a pagar pelos serviços completos.

\textbf{Plano Pago}

Nosso Plano Pago oferece aos usuários acesso ilimitado a todas as funcionalidades e recursos avançados da plataforma. Com este plano, os usuários podem fazer o upload de múltiplos editais e aproveitar a geração ilimitada de tarefas.

Para cobrir os custos operacionais e garantir a sustentabilidade da plataforma, estimamos o preço mensal da assinatura em R\$24,99 e o preço anual em R\$249,90. Essa estrutura de preços proporciona aos usuários flexibilidade e economia, incentivando-os a aderir ao plano anual para obter um desconto significativo.


\section{Prova de Conceito (POC)}

Na \ac{poc}, o objetivo é o teste da arquitetura e tecnologias propostas anteriormente, com a intenção de verificar se os componentes se integram e o sistema pensado é possível de ser desenvolvido.

\subsection{Mudanças}

Nosso objetivo ao desenvolver a \acs{poc} era testar e validar o funcionamento das tecnologias e da infraestrutura planejadas inicialmente. Para isso, implementamos as integrações essenciais utilizando o Firebase, como login e registro, além de uma integração básica com o serviço de pagamento Stripe. Também desenvolvemos o motor de processamento e identificação dos conteúdos e matérias de um edital dentro da plataforma, junto das funcionalidades no frontend para suportar esse recurso.


	

 


\section{Mínimo Produto Víavel (MVP)}

Esta seção tem por objetivo informar as alterações, descartes e escolhas feitas no projeto para o MVP, em relação ao sistema da POC. Essas mudanças refletem os problemas e inconsistências enfrentadas durante o desenvolvimento da POC, e foram motivadas pela necessidade de otimizar a eficiência e o desempenho do sistema, de forma a atender os requisitos do MVP da melhor maneira possível.

\subsection{Mudanças na Infraestrutura}
 
	A principal atualização feita no projeto em relação à POC foi o descarte do Google Gemini, substituído pelo serviço da OpenAI. Inicialmente, o Gemini foi escolhido devido ao seu baixo custo e à integração facilitada com o Firebase, visando manter todos os serviços em um único ecossistema. Além disso, o Gemini oferecia um serviço de treinamento de modelos customizados, baseado em outros modelos de processamento de texto pré-treinados da Google.
 
	Um dos primeiros problemas enfrentados com o serviço do Gemini foi justamente a customização dos modelos. Embora tivéssemos sucesso em treinar e utilizar um modelo com textos de editais e retorno de JSON, não conseguimos utilizar o mesmo fora da plataforma da Google, ou seja, não conseguimos integrar o serviço ao nosso código. Além disso, por ser um serviço relativamente recente, a documentação em certas funcionalidades ainda é escassa, assim como o suporte da comunidade.
 
	Os problemas que mais nos afetaram foram a inconsistência e os bloqueios ao utilizar um modelo de processamento de texto já treinado. Os textos JSON gerados pela IA eram frequentemente quebrados ou não formatados corretamente, impossibilitando seu consumo dentro da plataforma. Além disso, ao enviar o mesmo texto do edital diversas vezes para processamento, o sistema retornava erros de "recitação" ou de conteúdo não autorizado. Esses erros também possuíam uma documentação extremamente limitada, dificultando ainda mais a solução dos problemas.
 
	Essas dificuldades nos levaram a substituir o Google Gemini pelo serviço da OpenAI, que oferecia maior estabilidade e consistência na geração dos conteúdos, melhor documentação e suporte mais robusto. A substituição dos serviços foi um processo relativamente simples, e desde o início os modelos da OpenAI se mostraram muito eficientes. Porém, tivemos a mudança nos custos também, anteriormente, pois estávamos trabalhando com os serviços gratuitos do Gemini, e agora temos outros valores e custos consideravelmente maiores para levar em consideração para o projeto no futuro.
 
	A mudança do serviço de IA utilizado foi a única alteração relevante nesse período. O restante do sistema e outras integrações se mostraram satisfatórios, atendendo às expectativas e requisitos estabelecidos durante a fase de planejamento.
\chapter{Considerações Finais}

Ao longo do semestre, desenvolvemos diversas habilidades, tanto técnicas como comportamentais. O trabalho exigiu que tivessemos muita organização, além de exercitar o trabalho e a comunicação em grupo da equipe.

Desenvolver o Studyflow fez com o que o grupo deixasse a zona de conforto e trabalhasse com um tema fora do dia a dia, os concursos públicos. No processo de desenvolvimento da plataforma, muitas ideias de melhorias surgiram, que estão guardads para a implementação futura, sempre visando a melhor experiência do usuário final.

Com o auxílio da inteligência artificial e a critividade dos membros do grupo, o studyflow tem potencial de ser uma grande ferramenta para auxílio dos estudantes de concurso público e futuramente dos estudos que queiram prestar qualquer vestibular. Com todo o desenvolvimento e experimentação, ao fim atingimos o objetivo de entregar uma plataforma com potencial de mudar e ajudar vidas.

\begin{comment}

\chapter{Migrar do site}
% https://www.overleaf.com/learn/latex/Articles/How_to_write_in_Markdown_on_Overleaf
\todo[inline]{Coloquei como outro arquivo  pds.md para ficar mais facil de ajustar e remover o que precisar conforme formos ajustando}
\markdownInput{pds.md}

\end{comment}


% ----------------------------------------------------------
% Finaliza a parte no bookmark do PDF
% para que se inicie o bookmark na raiz
% e adiciona espaço de parte no Sumário
% ----------------------------------------------------------
\phantompart

% ----------------------------------------------------------
% ELEMENTOS PÓS-TEXTUAIS
% ----------------------------------------------------------
\postextual
% ----------------------------------------------------------

% ----------------------------------------------------------
% Referências bibliográficas
% ----------------------------------------------------------

\bibliography{referencias}

%\input{pos-glossario}

%\input{pos-apendices}

%\input{pos-anexos}


%---------------------------------------------------------------------

\end{document}