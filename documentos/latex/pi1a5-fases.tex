\chapter{Fases da disciplina}
\label{fases-da-disciplina}
Neste capítulo são apresentadas as fases e tarefas do projeto. Todas as atividades são avaliadas pelos professores durante o período da disciplina e não somente nas entregas. 
As datas de entregas podem ser encontradas no plano de aulas disponível no \ac{suap}.


\section{Primeira fase}
A primeira fase consiste nas etapas de definição da equipe, análise dos projetos e repositórios, elaboração a apresentação de uma proposta inicial de projeto. 

\begin{itemize}
    \item Definição da equipe \atividade{definicao-equipe};
    
    \item Definição do(a) gerente da equipe \atividade{escolha-gerente};
    
    \item Linha de definição dos acessos no mesmo modelo do acessos.txt do repositório \gls{svn} \atividade{acessos};
    
    \item Criação de blog \atividade{criacao-blog};
    
    \item Criação de canal no YouTube \atividade{criacao-youtube};
    
    \item Arquivo equipe.yaml validado com yamlint com as definições e urls do projeto, blog, YouTube etc \atividade{equipe-yaml};
    
    \item Planejamento \atividade{planejamento-projeto};
    
    \item Proposta inicial \atividade{proposta-inicial};
    
    \item Inicio do desenvolvimento do projeto.
\end{itemize}


\section{Segunda fase}
A segunda fase consiste na entrega da prova de conceito e apresentação do desenvolvimento até o momento.

\begin{itemize}
    \item Desenvolvimento do projeto;

    \item Prova de conceito \atividade{prova-de-conceito};
    
    \item Desenvolvimento do projeto;
\end{itemize}

\section{Terceira fase}
A terceira fase consiste na entrega e apresentação da primeira versão de um \ac{mvp}.

\begin{itemize}
    \item Entrega primeira versão \atividade{primeira-versao};

    \item Apresentação primeira versão \atividade{apresentacao-primeira-versao} para banca.
\end{itemize}

\section{Continuamente durante o projeto}

\begin{itemize}
    \item Publicação semanal no blog \atividade{publicacao-blog};
    
    \item A cada apresentação:
      \begin{itemize}
            \item Auto avaliação da equipe \atividade{planilha-notas};
          
            \item \emph{Slides} devem ser colocados no repositório de controle de versão;
          
            \item Vídeo do \gls{gource} \atividade{video-gource}.
      \end{itemize}

       
\end{itemize}



% % \subsubsection{Armadilhas e problemas}
% \subsubsection{Revisão Inicial da Literatura}


\todo[inline]{Uma coisa que sempre pega é deixarem para fazer coisas na ultima hora. Sempre falo no primeiro dia mas não adianta, alguma sugestão ? Gource, latexdiff são processos simples mas precisam baixar os programas e o conhecimento mínimo de executar um comando, muitas vezes ficam apanhando com coisas básicas como espaços nos nomes das pastas, não lerem as mensagens de erro etc....}
