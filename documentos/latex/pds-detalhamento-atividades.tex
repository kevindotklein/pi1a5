% Atividades exclusivas de PDS

\section{Pesquisa e avaliação de projetos  anteriores}\label{atv-avaliacao-projetos-anteriores}
Durante a definição do projeto a ser desenvolvido deve ser feita uma pesquisa em projetos anteriores disponíveis no repositório de controle de versão \atividade{controle-de-versao}, tanto do curso técnico integrado (\textbf{AXXXX*}) como do curso superior (\textbf{SXXXXYY*}) que se encontram no repositório.

Deverá ser feita Análise e Avaliação de dois projetos anteriores que utilizem preferencialmente o mesmo tipo de tecnologia que será utilizada no projeto da equipe deseja desenvolver.
\begin{itemize}
    \item detalhar quais pontos compatíveis levaram a escolha dos dois projetos analisados;
    
    \item gerar documento indicando problemas e melhorias;
    
    \item verificar os logs de commits do repositório de controle de versão;
    
    \item o mesmo projeto não poderá ser utilizado pelas equipes mais de uma vez para análise, portanto devem se organizar para evitar a duplicação;
    
    \justificativa{Duas equipes fizeram em conjunto a avaliação de um mesmo projeto}
    
    \item indicar o que aprenderam com os projetos, que cuidados devem tomar a partir da leitura dos outros projetos (tanto os dois analisados e documentados como outros que tenham lido antes de fazer a escolha final dos dois projetos);
    
    \item fazer análise crítica, não acreditar no que esta só no documento (verificar se o que esta no documento bate com a realidade que pode ser observada pelo blog, \glspl{commit}, arquivos do repositório ).
\end{itemize}


\section{Relatório com análise dos outros projetos}\label{atv-relatorio-analise-projetos}

Cada equipe deve assistir as apresentações das outras equipes e analisar o que foi entregue e apresentado gerando um relatório de avaliação indicando pontos negativos, pontos positivos, problemas, criticas, sugestões de melhorias para a entrega seguinte. É importante que cada equipe assista as outras apresentações e já trabalhe no desenvolvimento desse relatório que deve ser entregue ao final do período de apresentações.

O relatório deve ser enviado para o repositório de controle de versão \atividade{controle-de-versao}. Deve ser gerado um documento com uma análise geral de todos os outros projetos e também um documento especifico para cada outra equipe que apresentou. 
Os documentos devem colocados no repositório de controle de versão com o seguinte padrão:
\begin{itemize}
    \item \textbf{/Documentos/Analise\_Outros\_Projetos/Resumo.pdf};
    
    \item \textbf{/Documentos/Analise\_Outros\_Projetos/XX\_Nome\_do\_Projeto.pdf};
    
    \item Sendo \textbf{XX} a ordem de apresentação (01,02,03 ...) e \textbf{Nome\_do\_Projeto} o nome do projeto que foi analisado.
\end{itemize}

 Cada equipe deverá ler os relatórios das outras equipes e verificar a pertinência, possibilidade e efetuar os ajustes / melhorias além dos pontos apontados pelos professores na banca de apresentação \atividade{apresentacao-primeira-versao}.

Todos os ajustes devem ser apresentados na entrega final do projeto \atividade{entrega-final}.

\section{Entrega final}\label{atv-entrega-final}

Durante a primeira apresentação do projeto \atividade{apresentacao-primeira-versao} os membros da banca apresentam suas visões sobre o projeto e devolvem os documentos com anotações e as outras equipes fazem suas análises \atividade{relatorio-analise-projetos}. A equipe deve analisar todos as informações e fazer os ajustes necessários para uma nova entrega.


\begin{itemize}

    \item Apresentação de até 25 minutos das mudanças e correções que foram feitas;
    
    \item Relatório do desenvolvimento \atividade{relatorio-do-desenvolvimento} atualizado e documento \gls{latexdiff} - \atividade{latexdiff};

    \item Vídeo do \gls{gource} \atividade{video-gource};
    
    \item Novo vídeo de demonstração da aplicação;

    \item Auto avaliação da equipe \atividade{planilha-notas}.
    
\end{itemize}
