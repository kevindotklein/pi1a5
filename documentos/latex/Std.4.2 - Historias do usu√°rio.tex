\newpage
\section{Análise de Requisitos}

Neste tópico é analisado e definido fatores de negócio do projeto, como requisitos funcionais, requisitos não funcionais, regras de negócios e histórias de usuários. Todos esses fatores foram levados em conta para o desenvolvimento do projeto e devem ser seguidos de forma objetiva e concreta.

\subsection{Regras de Negócio}

As regras de negócio são definições importantes para qualquer sistema e especificam o que é o escopo do sistema. No Quadro \ref{tab:Regras de negócio} estão especificadas as regras de negócio do Studyflow.

\begin{quadro} [h]
 \caption{Regras de Negócio} \label{tab:Regras de negócio}
 
    \begin{tabular}{|c|c|}
        \hline
        \makecell{ID} & \makecell{Descrição}\\
        
        \hline
        \makecell{RN1} & \makecell{Ao acessar o site, para ter acesso aos conteúdos e funcionalidades,\\ usuário precisa estar logado}\\
        
        \hline
        \makecell{RN2} & \makecell{Apenas usuários autenticados podem executar as funcionalidades do sistema}\\

        \hline
        \makecell{RN3} & \makecell{Após o login bem-sucedido, o sistema deve redirecionar o usuário \\para a página principal da plataforma, onde ele terá acesso\\ às funcionalidades disponíveis, tais quais, upload do edital}\\

        \hline
        \makecell{RN4} & \makecell{O sistema terá upload de arquivos para envio do edital\\ ou um campo para digitar os conteúdos exigidos no edital}\\

        \hline
        \makecell{RN5} & \makecell{O sistema deve processar o arquivo enviado pelo usuário\\ e extrair os dados relevantes, como disciplinas, conteúdos e datas de provas}\\

        \hline
        \makecell{RN6} & \makecell{Caso o formato do arquivo do edital enviado não seja suportado\\ ou o conteúdo não seja identificado corretamente,\\ o sistema deve fornecer feedback ao usuário sobre o problema}\\

        \hline
        \makecell{RN7} & \makecell{O sistema terá um calendário semanal onde os cards serão organizados}\\

        \hline
        \makecell{RN8} & \makecell{A rotina de estudos gerada deve ser baseada nas informações \\extraídas do edital do concurso e nas preferências de estudo do usuário,\\ como disponibilidade de tempo e prioridades de aprendizado}\\

        \hline
        \makecell{RN9} & \makecell{Os cards semanais devem exibir informações detalhadas\\ sobre os temas de estudo, incluindo descrição da matéria e\\ tempo estimado de estudo}\\

        \hline
        \makecell{RN10} & \makecell{O usuário deve ser capaz de acessar a interface de edição dos \\cards semanais de tarefas a qualquer momento}\\

        \hline
        \makecell{RN11} & \makecell{As modificações feitas nos cards, como mover para outro dia da semana\\ ou alterar a descrição da tarefa, devem ser refletidas\\ instantaneamente na visualização da rotina de estudos}\\

        \hline
        \makecell{RN12} & \makecell{O sistema terá um plano pago onde as rotinas serão\\ geradas sem limite e o número de editais será maior}\\

        \hline

        
    \end{tabular}
    \fonte{Os autores.}
\end{quadro}


\subsection{Requisitos Funcionais}

Em Engenharia de Software, os requisitos funcionais são a espinha dorsal de qualquer projeto, definindo as funcionalidades essenciais que o sistema deve oferecer para atender às necessidades dos usuários e alcançar os objetivos do projeto. 
Segundo \cite{sommerville2011software}, "requisitos funcionais especificam o que o sistema deve fazer". Eles descrevem as tarefas e atividades que o software deve realizar, traduzindo as expectativas dos stakeholders em funcionalidades concretas.



Para o Studyflow, os seguintes requisitos funcionais estão descritos do Quadro \ref{tab:Requisitos Funcionais }.

\begin{quadro} [!ht]

     \centering
     
 	    \caption{Requisitos Funcionais} \label{tab:Requisitos Funcionais }

    \begin{tabular}{|c|c|}
        \hline
        \makecell{ID} & \makecell{Descrição}\\
    
        \hline
        \makecell{RF1} & \makecell{Sistema de cadastro e login}\\

        \hline
        \makecell{RF2} & \makecell{O sistema deve permitir que o usuário faça upload de\\ um arquivo contendo o edital do concurso desejado}\\

        \hline
        \makecell{RF3} & \makecell{O sistema deve permitir que o usuário digite o conteúdo\\ a ser estudado exigido pelo edital}\\

        \hline
        \makecell{RF4} & \makecell{O sistema deve processar o arquivo enviado\\ e extrair automaticamente as informações relevantes}\\

        \hline
        \makecell{RF5} & \makecell{O sistema deve fornecer uma interface para que\\ o usuário visualize sua rotina de estudos semanalmente}\\

        \hline
        \makecell{RF6} & \makecell{A visualização da rotina de estudos deve incluir\\ cards organizados por dia da semana, exibindo os temas de \\estudo, descrições das tarefas e tempos estimados de estudo}\\

        \hline
        \makecell{RF7} & \makecell{O sistema deve permitir que o usuário edite \\os cards de tarefas em sua rotina de estudos}\\

        \hline
        \makecell{RF8} & \makecell{A movimentação dos cards deve ser intuitiva e \\feita através de arrastar e soltar na interface da plataforma}\\

        \hline
        \makecell{RF9} & \makecell{O sistema deve solicitar feedback semanal ao \\usuário sobre os conteúdos estudados}\\

        \hline
        \makecell{RF10} & \makecell{O sistema deve ser capaz de gerar automaticamente uma\\ rotina de estudos com base nas informações extraídas \\do edital do concurso e nas preferências do usuário}\\

        \hline
        \makecell{RF11} & \makecell{O sistema deve permitir que o usuário personalize\\ suas preferências de estudo, como horários disponíveis\\ e prioridades de aprendizado}\\

        \hline
        \makecell{RF12} & \makecell{As preferências de estudo personalizadas devem ser\\ levadas em consideração na geração da rotina de estudos}\\
        
        \hline
        
    \end{tabular}
\end{quadro}

    
\subsection{Requisitos Não Funcionais}


Garantindo a Qualidade do Software, os requisitos não funcionais complementam os requisitos funcionais, definindo as características e qualidades que o sistema deve apresentar para garantir sua usabilidade, confiabilidade, segurança, desempenho e outras características essenciais. 
Para \cite{pressman2014software}, "requisitos não funcionais definem como o sistema deve se comportar". Eles detalham as restrições e os critérios que o software deve atender para oferecer uma experiência satisfatória aos usuários.

\begin{quadro} [!ht]
        \centering
     
 	    \caption{Requisitos Não Funcionais} \label{tab:Requisitos Não Funcionais }

    \begin{tabular}{|c|c|}
        \hline
        \makecell{ID} & \makecell{Descrição}\\

        \hline
        \makecell{RNF1} & \makecell{O sistema deve garantir a segurança dos dados do usuário,\\ implementando medidas de criptografia para proteger\\ informações sensíveis, como senhas e dados pessoais}\\

        \hline
        \makecell{RNF2} & \makecell{O acesso aos dados do usuário deve ser restrito apenas a\\ usuários autorizados e ser protegido contra acessos não\\ autorizados ou ataques maliciosos}\\

        \hline
        \makecell{RNF3} & \makecell{O sistema deve ser capaz de lidar com um grande volume de\\ usuários simultâneos sem degradar significativamente o desempenho}\\

        \hline
        \makecell{RNF4} & \makecell{O tempo de resposta do sistema para as solicitações do\\usuário deve ser no máximo 3 segundos, garantindo uma\\ experiência de uso fluida e sem atrasos perceptíveis}\\

        \hline
        \makecell{RNF5} & \makecell{A plataforma deve ser compatível com os principais navegadores\\ web, como Google Chrome, Mozilla Firefox, Safari e Microsoft Edge,\\ garantindo uma experiência consistente para todos os usuários}\\

        \hline
        \makecell{RNF6} & \makecell{A interface do usuário da plataforma deve ser responsiva e\\ adaptável a diferentes dispositivos e tamanhos de tela,\\ incluindo desktops, laptops, tablets e smartphones}\\

        \hline
        \makecell{RNF7} & \makecell{O sistema deve oferecer suporte para inglês e português}\\

        \hline
        
    \end{tabular}
\end{quadro}

\section{Histórias de Usuário} 

\begin{enumerate}[label=\textbf{\arabic*.}]
    
    \item \textbf{Realizar Cadastro} \newline
    
    Caso: Como usuário interessado em utilizar a plataforma de estudos, desejo me cadastrar no sistema para ter acesso aos recursos disponíveis.

    Critérios de Aceitação:
    \begin{itemize}
        \item O usuário deve conseguir preencher o formulário de cadastro.
        \item O e-mail fornecido pelo usuário deve ser único no sistema.
        \item A senha deve atender aos padrões de segurança estabelecidos pelo site.
    \end{itemize} 

    \newpage
    \item \textbf{Realizar \textit{Login}} \newline

    Caso: Como usuário registrado, desejo fazer login na plataforma para acessar os recursos disponíveis.

    Critérios de Aceitação:
    \begin{itemize}
        \item O usuário deve conseguir fazer login com sucesso utilizando e-mail e senha corretos.
        \item O acesso aos recursos da plataforma só deve ser permitido para usuários logados.
    \end{itemize}
    
    \item \textbf{Upload de Edital para Concurso} \newline
    
    Caso: Como usuário da plataforma de estudos, desejo poder fazer o upload de um edital de concurso para que a plataforma possa gerar uma rotina de estudos personalizada.

    Critérios de Aceitação:
    \begin{itemize}
        \item O usuário deve conseguir fazer o upload de um arquivo contendo o edital do concurso desejado.
        \item A plataforma deve ser capaz de processar o edital e extrair os conteúdos relevantes para a preparação.
        \item Os conteúdos extraídos devem ser organizados de forma a criar uma rotina de estudos semanal.
    \end{itemize}

    \item \textbf{Gerar Rotina Semanal de Estudos} \newline

    Caso: Como usuário da plataforma, desejo que a plataforma gere uma rotina semanal de estudos com base nos conteúdos do edital do concurso que fiz o upload.

   Critérios de Aceitação:
   \begin{itemize}
       \item A plataforma deve analisar o edital e propor uma rotina de estudos que cubra todos os temas e disciplinas exigidos.
       \item A rotina deve ser organizada em cards em um calendário semanal, indicando os temas a serem estudados a cada dia.
       \item Os cards devem ser movíveis para permitir ao usuário ajustar a rotina de acordo com sua disponibilidade e preferências.
   \end{itemize}

   \item \textbf{Coleta de \textit{Feedback} Semanal} \newline

   Caso: Como usuário da plataforma de estudos, desejo fornecer \textit{feedback} semanal sobre os conteúdos estudados para que a plataforma possa ajustar a rotina de estudos.

   Critérios de Aceitação:
   \begin{itemize}
       \item Ao final de cada semana, o usuário deve ser solicitado a fornecer feedback sobre os conteúdos estudados.
       \item O feedback deve incluir informações sobre as matérias estudadas e a dificuldade encontrada em cada uma.
       \item A plataforma deve usar o feedback fornecido pelo usuário para ajustar a rotina de estudos para a próxima semana, priorizando as matérias com maior dificuldade.
   \end{itemize}
    
\end{enumerate}
