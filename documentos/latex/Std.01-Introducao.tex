\chapter{Introdução}

\justificativa{No Brasil, temos um importante instrumento de ascensão social através dos estudos, chamado “Concurso Público”. Muitas pessoas se dedicam para essa prova, pois através dela pode-se ter melhores ganhos e estabilidade financeira, além de depender do concurso um emprego sem demissão e plano de carreira atrativo. Com esse cenário, milhares de brasileiros tentam todos os anos a prova para esses concursos públicos para diferentes cargos e diversas esferas do poder público, mas com toda essa gama de provas e quantidade de concorrentes, como se preparar da melhor forma?}

Em nosso país, temos um mercado com grande potencial de crescimento e quando falamos de serviços para “concurseiros” (pessoas que prestam concursos públicos). Os serviços que tem a maior atratividade e público, são as plataformas de ensino direcionado para esse tipo de público. Com a Pandemia de COVID-19 iniciada em 2020, essas plataformas tiveram uma mudança em seu modelo de negócio, precisando se adaptar ao modelo online de educação ou EAD. 

Essa mudança no modelo de negócios das empresas do nicho de educação para concurso público, impactou os seus clientes, que agora consomem em maior quantidade serviços de educação online, aumentando ainda mais a possibilidade de novos serviços para esse tipo de cliente.

Com a migração de cursos antes no presencial, para agora online, os estudantes economizam tempo de deslocamento até a escola, tendo o serviço de aula na palma das mãos. E o tempo é justamente o ativo mais difícil para um estudante de concurso público, pois ele precisa conciliar suas atividades cotidianas e ainda seguir uma rotina de estudos e para iniciantes nesse mundo de concursos, além de tempo o que e como estudar se torna uma dificuldade maior ainda.

A tecnologia é uma grande aliada dos estudantes para ajudar a administrar esse cenário. Com o avanço de tecnologias de Inteligência Artificial, o que antes era um estudo e uma rotina sem parâmetros, pode ser otimizado e organizado de maneira muito mais fácil com a chegada desse recurso.


%Para criar uma sessão
\section{Objetivo}

Este projeto tem como objetivo ajudar os estudantes de concurso público a se organizarem com as matérias e rotina de estudos para um concurso público com o auxĺlio de tecnologias de Inteligência Artificial.

A plataforma em ambiente WEB permitirá que os estudantes enviem o edital em o qual irão prestar a prova e a partir desse edital a plataforma gere uma rotina de estudos personalizada pensando no maior ganho em sua jornada de estudos.

\end{itemize}

\section{Princípios norteadores}

\begin{itemize}
    \item Estimular a autonomia para pesquisa, organização das informações e tomada de decisão;
    
    \item Emular o mundo do trabalho e as dinâmicas do trabalho em equipe;
    
    \item Liberdade de escolha de metodologias, plataformas e tecnologias, desde que justificadas e previamente aceitas pelos professores da disciplina;
    
    \item Oferecer um ambiente seguro para experimentação, erro e aprendizado.
    
\end{itemize}

\todo[inline]{De alguma forma podemos indicar aqui que os projetos podem ser integrados com outras disciplinas desde que os professores concordem e que as avaliações são efetuadas de forma independente a partir dos critérios de cada disciplina}

\section{Visão geral do projeto}

A aplicação a ser desenvolvida deve resolver um problema real, pode ser uma aplicação nova ou também uma evolução ou módulo para uma aplicação já existente de código aberto ou que esteja no repositório de controle de versão.

Essa aplicação pode ser desenvolvida em qualquer plataforma ou linguagem de desenvolvimento desde que compatível com o objetivo (ex uma aplicação para controle de lista de compras não pode ser desenvolvida somente para desktop) e os requisitos apresentados neste documento. A aplicação deve atender aos requisitos apresentados no \autoref{requisitos-aplicacoes}.

\todo[inline]{Precisamos deixar claro aqui a questão dos processos}



\section{Papéis e responsabilidades}

Na disciplina existem diversos papéis que são assumidos pelos participantes, o entendimento desses papéis permite atingir corretamente os objetivos da disciplina:
\begin{itemize}
    \item \textbf{Estudante} - Deve desenvolver as atividades da disciplina seguindo os preceitos deste documento e orientações passadas pelos professores, colaborando para o sucesso do projeto desenvolvido pela equipe;


    \item \textbf{Equipe} - Segundo \cite{EQUIPES}: 
    \begin{citacao}“Um grupo de pessoas com alto grau de interdependência está direcionado para a realização de uma meta ou para a conclusão de uma tarefa, cria-se o conceito de EQUIPE. Em outra palavras, membros de uma equipe concordam com uma meta e concordam que a única maneira de alcançar essa meta é trabalhar em conjunto". 
    \end{citacao}
    Desta forma, as equipes são compostas por um número definido de estudantes, que tem como objetivo a concretização do trabalho da disciplina.
    
    Algumas outras definições de equipes e vídeos de apoio podem ser encontrados em: \dicasIvan{equipes}
    

    \item \textbf{Professor} - Tem o papel de orientar e avaliar, buscando atingir os objetivos da disciplina;
    
    \justificativa{Eles precisam buscar informações antes de nos questionar e vamos atender de acordo com requisições, não devemos influenciar nas decisões se não formos questionados ou consultados}
    

    \item \textbf{Cliente} - Os professores assumem o papel de cliente do projeto e devem ser consultados como um cliente real. Ao desenvolver uma aplicação para resolver um problema real e tendo acesso a usuários reais o projeto pode evoluir muito pois passa por diversas visões em relação ao problema. Uma equipe também pode assumir o papel de cliente de outra equipe desde que não entrem em conflito com as decisões dos clientes principais que são os Professores;

    \item \textbf{Banca Examinadora} - O trabalho é apresentado para uma banca que vai avaliar tanto os documentos demonstrando o desenvolvimento como a aplicação em execução. Essa banca é composta pelos professores da disciplina e convidado(s).
    
\end{itemize}


\section{Avaliações}

Todos as atividades desenvolvidas durante o projeto são avaliadas pelos professores, prazos de entregas são considerados como datas de Provas / Avaliações tradicionais e portanto devem ser respeitados.

Além da documentação e da execução da aplicação, são avaliados os modelos estáticos e dinâmicos da aplicação, bem com a relação entre modelos, aplicação e objetivos do projeto.

A divisão das atividades desempenhadas por cada elemento da equipe deve ser documentada no trabalho. A avaliação pode ser individualizada, conforme as atividades desempenhadas por cada aluno ao longo do projeto.

Cada equipe também faz avaliações de seus membros, cada participante avalia todos os membros de sua equipe (incluindo ele mesmo). Essas avaliações também poderão ser consideradas pelos professores ao definir a nota individual de cada participante.




\section{Comunicação entre alunos e professores}
Existem diversos canais de comunicação que poderão ser utilizados durante o desenvolvimento da disciplina, conforme a seguir descrito.
\begin{itemize}
    \item Comunicador do \ac{suap} onde os professores muitas vezes enviam comunicados oficiais que precisam de registro;
    
    \item Curso definido no Moodle da disciplina onde algumas atividades devem ser entregues e também permite o envio de mensagens;
    
    \item E-mail dos alunos para os professores com dúvidas especificas (sempre enviar com cópia para ambos professores e indicar claramente qual a turma/equipe que faz parte, pois os professores tem projetos de diversas turmas e nem sempre os e-mails chegam com o nome correto do aluno);
    
    \item Grupo no Telegram que permite a comunicação entre todos participantes da turma, dúvidas genéricas devem ser feitas principalmente por esse canal pois permitem que todos tenham acesso a informação. Nesse grupo são enviados os links para as aulas/conversas síncronas da disciplina;
    
    \item Ferramentas de conferencia (Meet, RNP, Teams, Telegram  etc) para conversas síncronas.
    
\end{itemize}

É importante lembrar que algumas ferramentas devem ser utilizadas com cuidado, não se deve enviar mensagem com notificação de madrugada por exemplo. No Telegram é possível agendar o envio de uma mensagem ou até enviar sem a notificação, bastando escolher isso no momento do envio.

Existe um canal genérico de projetos (IFSP-SPO-Projetos) onde algumas informações gerais que atendem alunos do ensino médio e superior são publicadas \url{https://t.me/joinchat/AAAAAET-oEt6v2nyhgx2CQ}.





\justificativa{O Telegram não dá acesso ao número de telefone se o usuário não desejar, assim respeita um pouco o que a própria escola faz onde não temos acesso aos números, os grupos tem o histórico disponível e assim os alunos tem acesso ao que já aconteceu}




