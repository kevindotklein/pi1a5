\chapter{Introdução}

\justificativa{No Brasil, temos um importante instrumento de ascensão social através dos estudos, chamado “Concurso Público”. Muitas pessoas se dedicam para essa prova, pois através dela pode-se ter melhores ganhos e estabilidade financeira, além de depender do concurso um emprego sem demissão e plano de carreira atrativo. Com esse cenário, milhares de brasileiros tentam todos os anos a prova para esses concursos públicos para diferentes cargos e diversas esferas do poder público, mas com toda essa gama de provas e quantidade de concorrentes, como se preparar da melhor forma?}

Em nosso país, temos um mercado com grande potencial de crescimento e quando falamos de serviços para “concurseiros” (pessoas que prestam concursos públicos). Os serviços que tem a maior atratividade e público, são as plataformas de ensino direcionado para esse tipo de público. Com a Pandemia de COVID-19 iniciada em 2020, essas plataformas tiveram uma mudança em seu modelo de negócio, precisando se adaptar ao modelo online de educação ou EAD. 

Essa mudança no modelo de negócios das empresas do nicho de educação para concurso público, impactou os seus clientes, que agora consomem em maior quantidade serviços de educação online, aumentando ainda mais a possibilidade de novos serviços para esse tipo de cliente.

Com a migração de cursos antes no presencial, para agora online, os estudantes economizam tempo de deslocamento até a escola, tendo o serviço de aula na palma das mãos. E o tempo é justamente o ativo mais difícil para um estudante de concurso público, pois ele precisa conciliar suas atividades cotidianas e ainda seguir uma rotina de estudos e para iniciantes nesse mundo de concursos, além de tempo o que e como estudar se torna uma dificuldade maior ainda.

A tecnologia é uma grande aliada dos estudantes para ajudar a administrar esse cenário. Com o avanço de tecnologias de Inteligência Artificial, o que antes era um estudo e uma rotina sem parâmetros, pode ser otimizado e organizado de maneira muito mais fácil com a chegada desse recurso.


%Para criar uma sessão
\section{Objetivo}

Este projeto tem como objetivo ajudar os estudantes de concurso público a se organizarem com as matérias e rotina de estudos para um concurso público com o auxĺlio de tecnologias de Inteligência Artificial.

A plataforma em ambiente WEB permitirá que os estudantes enviem o edital em o qual irão prestar a prova e a partir desse edital a plataforma gere uma rotina de estudos personalizada pensando no maior ganho em sua jornada de estudos.

\end{itemize}

\section{Princípios norteadores}

\begin{itemize}
    \item Estimular a autonomia para pesquisa, organização das informações e tomada de decisão;
    
    \item Emular o mundo do trabalho e as dinâmicas do trabalho em equipe;
    
    \item Liberdade de escolha de metodologias, plataformas e tecnologias, desde que justificadas e previamente aceitas pelos professores da disciplina;
    
    \item Oferecer um ambiente seguro para experimentação, erro e aprendizado.
    
\end{itemize}

\todo[inline]{De alguma forma podemos indicar aqui que os projetos podem ser integrados com outras disciplinas desde que os professores concordem e que as avaliações são efetuadas de forma independente a partir dos critérios de cada disciplina}

\section{Visão geral do projeto}

A aplicação a ser desenvolvida deve resolver um problema real, pode ser uma aplicação nova ou também uma evolução ou módulo para uma aplicação já existente de código aberto ou que esteja no repositório de controle de versão.

Essa aplicação pode ser desenvolvida em qualquer plataforma ou linguagem de desenvolvimento desde que compatível com o objetivo (ex uma aplicação para controle de lista de compras não pode ser desenvolvida somente para desktop) e os requisitos apresentados neste documento. A aplicação deve atender aos requisitos apresentados no \autoref{requisitos-aplicacoes}.

\todo[inline]{Precisamos deixar claro aqui a questão dos processos}



\input{geral-introducao-papeis}
\input{geral-introducao-avaliacoes}
\input{geral-introducao-comunicacao}

