\chapter{Revisão da Literatura}


A seção de revisão da literatura tem como objetivo apresentar pesquisas, artigos, livros ou afins já realizados por outros autores, que embasem as problematizações utilizadas como motivo para o desenvolvimento da aplicação StudyFlow. Isto é, serão abordadas fontes de informação que comprovam a utilidade e pertinência da aplicação.

\section{Dificuldade em organizar rotina para estudos}

Com o crescimento exponencial da internet, cada vez mais informações são disponibilizadas ao público. No entanto, para aqueles que buscam conteúdo específico para estudar para concursos públicos, muitos dos resultados apresentados em uma busca regular em mecanismos de pesquisa podem ser irrelevantes ou incompletos.

Além da dificuldade em encontrar o conteúdo relevante para os concursos, os estudantes enfrentam desafios adicionais ao tentar organizar todas as informações. Esta é uma das maiores barreiras para aqueles que ainda não adotaram nenhum método ou rotina de estudos. Lia Salgado, autora do livro "Como vencer a maratona dos concursos públicos", destaca: "Esse é o primeiro impacto, mesmo. É assustador. Eu senti isso na pele quando comecei a minha preparação." sobre o volume de conteúdo a ser estudado para concursos. "A solução é organizar o estudo, planejar a rotina diária para ter o momento certo de estudar e distribuir as matérias ao longo da semana."

\section{A importância da Inteligência Artificial no apoio dos estudos}

A \ac{ia} está revolucionando diversos setores da sociedade, e a educação não é exceção. Nos últimos anos, ferramentas e plataformas impulsionadas por IA vêm surgindo com o objetivo de auxiliar os alunos em sua jornada de aprendizado, tornando-a mais personalizada, eficiente e eficaz.


Em destaque, os principais benefícios da \acs{ia} na educação é a sua capacidade de personalizar o ensino de acordo com as necessidades individuais de cada aluno. Através de algoritmos de aprendizado de máquina, os sistemas de \acs{ia} podem analisar o desempenho, estilo de aprendizado e ritmo de cada estudante, adaptando o conteúdo, as atividades e os métodos de ensino de forma otimizada \cite{shemshack2020systematic}.

A utilização de ferramentas impulsionadas por \ac{ia} não é novidade no mercado, mas a partir de 2022 houve um boom nesse mercado com a chegada do  ChatGPT, criado pela OpenAI. Para fins estudantis, os chatbots são as ferramentas amplamente adotadas, por sua facilidade de desenvoltura e compreensão pelo usuário.
Os Chatbots podem responder a dúvidas sobre a matéria, sistemas de tutoria oferecem exercícios personalizados e plataformas de aprendizado adaptativo sugerem conteúdos relevantes para cada estudante \cite{silva2023inteligencia}. Além disso, ferramentas de reconhecimento de fala e texto podem auxiliar alunos com deficiências e softwares de tradução podem facilitar o aprendizado em diferentes idiomas.




