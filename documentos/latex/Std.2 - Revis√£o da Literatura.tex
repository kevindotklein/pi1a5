\chapter{Revisão da Literatura}


A seção de revisão da literatura tem como objetivo apresentar pesquisas, artigos, livros ou afins já realizados por outros autores, que embasem as problematizações utilizadas como motivo para o desenvolvimento da aplicação StudyFlow. Isto é, serão abordadas fontes de informação que comprovam a utilidade e pertinência da aplicação.

\section{Dificuldade em organizar rotina para estudos}

Com o crescimento exponencial da internet, cada vez mais informações são disponibilizadas ao público. No entanto, para aqueles que buscam conteúdo específico para estudar para concursos públicos, muitos dos resultados apresentados em uma busca regular em mecanismos de pesquisa podem ser irrelevantes ou incompletos.

Além da dificuldade em encontrar o conteúdo relevante para os concursos, os estudantes enfrentam desafios adicionais ao tentar organizar todas as informações. Esta é uma das maiores barreiras para aqueles que ainda não adotaram nenhum método ou rotina de estudos. Lia Salgado, autora do livro "Como vencer a maratona dos concursos públicos", destaca: "Esse é o primeiro impacto, mesmo. É assustador. Eu senti isso na pele quando comecei a minha preparação." sobre o volume de conteúdo a ser estudado para concursos. "A solução é organizar o estudo, planejar a rotina diária para ter o momento certo de estudar e distribuir as matérias ao longo da semana."


\section{O papel da tecnologia na preparação para concursos públicos}

Com o auxílio de tecnologias que facilitem a organização da rotina do estudante, ele poderá focar em absorver o conteúdo passado e revisá-lo se necessário. Para maximizar a eficiência na geração das rotinas de estudo, a aplicação contará com inteligência artificial, que entenderá os conteúdos abordados no edital do concurso e os estruturará na rotina do estudante.

A inteligência artificial não é apenas uma ferramenta para resultados rápidos, e pode ser uma parceira quando se trata de facilitar diversas tarefas repetitivas ou cansativas. O fato é que a inteligência artificial está cada vez mais sendo utilizada por estudantes universitários, por exemplo, segundo a CNN (2023).
