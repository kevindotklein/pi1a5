\chapter{Requisitos de gerenciamento dos projetos}

\todo[inline]{Talvez seja interessante fazer aqui algo parecido com o que foi feito no capitulo 4 \newline, talvez até unificar com o capitulo anterior}

Cada equipe tem liberdade para escolher a metodologia que vai utilizar, mas essas metodologias devem ser seguidas corretamente (ex não tem como utilizar Scrum sem utilizar \emph{sprint}).

O objetivo de uma metodologia de gerenciamento de projetos é, além de organizar o trabalho a ser feito, permitir que haja previsibilidade no que será feito, ou seja, ao surgimento de uma nova demanda ou na ocorrência de um imprevisto (como algum problema com um membro da equipe, ou o aparecimento de um requisito não previsto inicialmente), seja possível que os impactos de tal ocorrência sejam identificados e as devidas medidas para mitigação de seus efeitos sejam tomadas, seja pela reorganização do que precisa ser feito, seja pela determinação de ajustes de escopo do projeto.

Ainda, é possível também que, usando uma determinada metodologia como base, ajustes sejam feitos para que a metodologia se adéque ao dia-a-dia da equipe. Porém é importante que os ajustes feitos sejam devidamente explicados (e documentados) e que tais ajustes não impactem os objetivos na adoção de uma de gerenciamento de projetos.

Independente da metodologia escolhida e dos nomes utilizados em cada uma é importante que exista organização e os artefatos correspondentes de cada metodologia sejam utilizados, como:
\begin{itemize}
    \item Atas;
    
    \item Definição de Papéis e Responsabilidades;
    
    \item Cronogramas;
    
    \item Registro e acompanhamento de métricas;
    
    \item Reuniões.
\end{itemize}

Todos documentos devem ser sempre entregues em formato \ac{pdf} e gerados a partir do \LaTeX.















