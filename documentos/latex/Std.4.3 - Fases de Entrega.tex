\section{Fases de Entrega}

As fases de entrega de um projeto são etapas fundamentais que conduzem o desenvolvimento de uma solução desde sua concepção até sua implementação e lançamento. Cada fase representa um marco importante no ciclo de vida do projeto, onde objetivos específicos são alcançados e progresso significativo é realizado.

Nossas fases de entrega planejadas são:

\begin{itemize}
\item Planejamento e Análise:
Nesta fase, são identificados os requisitos do projeto, definidos os objetivos e escopo, e elaborado o plano de projeto detalhado. Também é realizada uma análise de viabilidade técnica e financeira.

\item Prova de Conceito (PoC):

A \ac{poc} é uma fase inicial do projeto na qual são desenvolvidos protótipos ou demonstrações que validam a viabilidade técnica das principais funcionalidades da plataforma. Nesta fase, focamos em implementar um conjunto mínimo de recursos para validar a solução proposta e demonstrar sua viabilidade.

\item Desenvolvimento Iterativo:
Após o sucesso da \acs{poc}, o desenvolvimento da plataforma começa em etapas iterativas. Funcionalidades adicionais são implementadas em ciclos de desenvolvimento curtos, permitindo feedback contínuo e ajustes conforme necessário.

\item Testes e Qualidade:
Durante todo o processo de desenvolvimento, são realizados testes rigorosos para garantir que a plataforma atenda aos requisitos de qualidade e segurança. Isso inclui testes de unidade, integração, aceitação do usuário e segurança.

\item Implantação e Lançamento:
Após a conclusão do desenvolvimento e dos testes, a plataforma é implantada em um ambiente de produção e está pronta para ser lançada. Isso pode incluir a configuração de servidores, migração de dados e treinamento de usuários.

\item Monitoramento e Manutenção:
Após o lançamento, a plataforma é continuamente monitorada para garantir que esteja funcionando conforme o esperado. Também são feitas atualizações regulares e manutenção para corrigir bugs, adicionar novos recursos e melhorar a experiência do usuário.

\end{itemize}

\textbf{Prova de Conceito (PoC) Inicial:}
Na \acs{poc} inicial, nosso foco seria desenvolver um protótipo funcional da plataforma que demonstre as principais funcionalidades propostas. Isso pode incluir a autenticação de usuários, a criação de tarefas e a integração com serviços de terceiros, como o Firebase para o back-end e a Vercel para o front-end. A \acs{poc} permite validar a viabilidade técnica da solução proposta e identificar possíveis desafios ou obstáculos que precisam ser superados antes da implementação completa.

\textbf{Entrega Final da Plataforma:}

Na entrega final da plataforma, todos os recursos e funcionalidades planejados são implementados e testados completamente. Isso inclui a implementação completa e treinamento dos modelos de \acs{ia}, adequação a políticas de segurança, a integração com serviços de terceiros, como autenticação, armazenamento de dados e hospedagem, e a garantia de que a plataforma atenda aos requisitos de desempenho, escalabilidade e usabilidade. A entrega final marca o lançamento oficial da plataforma e seu uso pelos usuários finais.