\chapter{GESTÃO DO PROJETO}

Nesta seção, serão apresentados os métodos escolhidos para a gestão do projeto e da equipe, com o objetivo de assegurar a melhor utilização possível do tempo, orçamento e recursos voltados para o projeto, para que esse possa ser concluído dentro do prazo estabelecido. Também serão levantados alguns riscos possíveis, afim de que com o conhecimento dessas possibilidades, medidas possam ser tomadas para evitá-los.  

\section{Formação da equipe}

A equipe foi formalizada durante as aulas da disciplina, porém já havia sido definida posteriormente. Todos os integrantes são alunos do curso de Tecnologia em Análise e Desenvolvimento de Sistemas, \acs{ifsp}, campus São Paulo. A equipe se baseia nos conhecimentos de cada um de seus membros com o objetivo de preencher as necessidades do projeto. Os participantes da equipe Noz são: 

\begin{itemize}
    \item \textbf{Kevin Klein}
    \item \textbf{Luiz Fernando Cavalcante de Faria}
    \item \textbf{Pedro Felipe da Silva Dias}
    \item \textbf{Ruan de Souza Cardoso Brito}
\end{itemize}

\subsection{Papéis}

Os papéis foram definidos através das habilidades dos membros , para que cada um pudesse atuar de maneira segura com seus conhecimentos, dessa maneira a organização e a fluidez do projeto são beneficiadas. As atividades são planejadas para que todos sejam responsáveis por alguma parte específica do projeto, podendo receber ajuda dos outros participantes caso seja necessário.

\subsection{Organização da atividades}

Dentro das atividades do projeto, a divisão está descrita no Quadro \ref{tab:Desenvolvimento }.
    \begin{quadro} [h]
    \centering
        \caption{Atividades de Desenvolvimento}\label{tab:Desenvolvimento }
 	\begin{tabular}{|c|c|c|c|c|}
        \hline
\textbf{Atividades}&\textbf{Kevin}&\textbf{Luiz}&\textbf{Pedro}&\textbf{Ruan} \\ \hline
Front-end &X & & &X \\ \hline
Back-end &X & &X & \\ \hline
Dados & &X & & \\ \hline
UX/UI & &X & & \\ \hline
Documentação & &X & &X \\ \hline
Inteligência Artificial &X & &X &\\ \hline
\end{tabular}
         \fonte{Os autores.}
    \end{quadro}


Além das atividades de desenvolvimento, as tarefas de gestão e planejamento foram divididas conforme o Quadro \ref{tab:Gestão }



    \begin{quadro} [h]
            \centering
 	    \caption{Atividades de gestão e planejamento} \label{tab:Gestão }



        \begin{tabular}{|c|c|c|c|c|}
    \hline
    \textbf{Atividades}&\textbf{Kevin}&\textbf{Luiz}&\textbf{Pedro}&\textbf{Ruan} \\    \hline
    SVN &X & & & \\ \hline
    LaTex & &X & & \\  \hline
    Blog & & & &X \\ \hline
    Youtube & & & &X \\ \hline
    Contato Orientador & & &X & \\ \hline
    Kanbam &X & & & \\ \hline
    Apresentações & &X & & \\ \hline
        \end{tabular}
            \fonte{Os autores.}
    \end{quadro}


\section{Gestão de tempo e desenvolvimento}

A equipe decidiu aderir à utilização do Scrum como \textit{framework} de gerenciamento, a fim de melhorar a organização durante o desenvolvimento do projeto. Ele foi escolhido pela sua eficiência e ampla utilização por diversas empresas no mercado, além de ser conhecido pelos integrantes do grupo, o que facilita sua implementação.  Além disso também optamos pela utilização do Kanban, para garantir a eficiência na realização das tarefas e o cumprimento dos prazos.

\subsection{Scrum}

O \textit{Scrum} é uma metodologia de desenvolvimento ágil amplamente empregada para lidar com a complexidade na criação de produtos. Este método valoriza a colaboração, a autonomia da equipe e a entrega progressiva e iterativa. Composto por uma série de práticas, papéis e artefatos, o Scrum promove a eficácia e a qualidade do trabalho realizado, impulsionando a entrega de valor de forma consistente ao longo do tempo.

\subsection{Kanban}

O Kanban é uma metodologia de gestão visual que teve origem no Japão e ganhou popularidade em diversos setores, como desenvolvimento de software, manufatura e serviços. O termo "Kanban" significa "sinal visual" em japonês, e essa abordagem se baseia na utilização de cartões ou post-its para representar unidades de trabalho e visualizar o fluxo do processo. Essa metodologia visa proporcionar transparência sobre o trabalho em andamento e controlar o \ac{wip} para otimizar a eficiência do sistema.

\section{Gestão de comunicação}
A comunicação é parte essencial para que tudo corra bem no projeto. Foram utilizados alguns meios para realizar esse diálogo entre a equipe.

O meios de comunicação utilizados internamente foram o  Whatsapp, aplicativo de mensagens instantâneas  e chamadas de voz, foi usado para troca de mensagens durantes as semanas, a fim de proporcionar agilidade e facilidade na comunicação, e o Discord, que é uma aplicação voltada para a comunicação, principalmente de grupos e comunidades, foi usado para as reuniões realizadas semanalmente.

Para a comunicação com o público foi criado um blog, na plataforma Blogger, onde são compartilhadas as atualizações semanais e informações relevantes sobre o projeto.

 \begin{figure}[!htb]
 	    \centering
 	    \caption{\label{logo}QR Code do Blog}
 	    \includegraphics[width=5cm]{img/qrcode-blog.png}
 	    \fonte{Os autores.}
\end{figure}
\FloatBarrier

Além do blog, é possível acompanhar o desenvolvimento do projeto e das apresentações pelos vídeos presentes no canal do Youtube dedicado ao projeto.

 \begin{figure}[!htb]
 	    \centering
 	    \caption{\label{logo}QRCode do Youtube}
 	    \includegraphics[width=5cm]{img/qrcode-youtube.png}
 	    \fonte{Os autores.}
\end{figure}
\FloatBarrier

\section{Análise de riscos}

Em todo projeto, riscos de todos os tipos e magnitudes podem ocorrer, impactando de forma significativa o andamento do projeto. 
Para a cosntrução do Studyflow, foram mapeados os riscos, o nível de impacto e a resposta aos mesmos, todos eles descritos no Quadro \ref{tab: Riscos}.

\begin{quadro} [h] % -- H posiciona exatamente no local em que foi inserido
    \centering
    \caption{Análise de Riscos}\label{tab: Riscos}
      
       \begin{tabular}{|c|c|c|}
       \hline
Risco &Nivel de Impacto &Resposta \\ \hline
Desistência pessoal &Alto &Aceitar \\ \hline
Problemas de Saúde &Alto &Aceitar \\ \hline
Conflitos Interpessoais &Médio &Mitigar \\ \hline
Comprometimento com outras tarefas &Médio &Mitigar \\ \hline
Mudança de Requisitos &Alto &Eliminar \\ \hline
Falhas de comunicação &Alto &Eliminar \\ \hline
Falta de conhecimento técnico &Médio &Mitigar \\ \hline
Problemas com conexão de rede &Baixo &Aceitar \\ \hline
Problemas com Hardware &Baixo &Aceitar \\ \hline
Escopo mal definido &Alto &Mitigar \\ \hline
Desempenho insastisfatório &Médio &Mitigar \\ \hline
Falhas de segurança &Médio &Mitigar \\ \hline
Falha em tecnologias externas &Médio &Aceitar \\ \hline
Problemas com o modelo de IA &Alto &Eliminar \\ \hline

\end{tabular}

\fonte{Os autores.}
\end{quadro}


Segue abaixo uma breve explicação sobre cada um dos possíveis riscos ao projeto:

\begin{itemize}
    \item \textbf{Desistência pessoal:} Membros da equipe abandonam o projeto, causando lacunas na expertise e sobrecarregando os membros restantes.
    \item \textbf{Problemas de saúde:} Membros da equipe enfrentam problemas de saúde que afetam sua capacidade de contribuir para o projeto.
    \item \textbf{Conflitos interpessoais:} Desentendimentos ou tensões entre membros da equipe prejudicam a colaboração e a eficiência do projeto.
    \item \textbf{Comprometimento com outras tarefas:} Membros da equipe têm prioridades divididas entre várias tarefas ou projetos, resultando em atrasos ou falta de dedicação ao projeto em questão.
    \item \textbf{Mudança de requisitos:} Alterações nos requisitos do projeto após o início do desenvolvimento, levando a retrabalho e atrasos.
    \item \textbf{Falhas na comunicação:} Comunicação inadequada entre membros da equipe, clientes ou partes interessadas, levando a mal-entendidos e erros.
    \item \textbf{Falta de conhecimento técnico:} Membros da equipe não possuem as habilidades ou conhecimentos necessários para concluir com sucesso determinadas tarefas ou aspectos do projeto.
    \item \textbf{Problemas com conexão de rede:} Problemas com a conexão de rede afetam a colaboração remota ou o acesso a recursos necessários para o projeto.
    \item \textbf{Falhas de hardware:} \textit{Hardware} essencial para o projeto falha, causando interrupções no desenvolvimento ou perda de dados.
    \item \textbf{Escopo mal definido:} Requisitos do projeto não estão claramente definidos desde o início, levando a confusão e revisões frequentes.
    \item \textbf{Desempenho insatisfatório:} O produto final não atende às expectativas de desempenho dos usuários, levando à insatisfação e possível rejeição.
    \item \textbf{Falha de segurança:} Vulnerabilidades de segurança no sistema comprometem a integridade ou a confidencialidade dos dados, resultando em riscos para os usuários e para a empresa.
    \item \textbf{Falha em tecnologias externas:} Dependência de tecnologias externas que podem falhar ou não atender às expectativas, afetando o desenvolvimento ou o funcionamento da aplicação.
    \item \textbf{Problema no treinamento da \acs{ia}:} Dificuldades no treinamento de sistemas de inteligência artificial para alcançar os resultados desejados, resultando em desempenho inadequado ou inexato.
\end{itemize}

